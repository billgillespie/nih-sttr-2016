\documentclass[11pt]{article}

% Use Helvetica font.
\usepackage{helvet}
\renewcommand{\familydefault}{\sfdefault}

% Use 1/2-inch margins.
\usepackage[margin=0.5in]{geometry}

\usepackage{natbib}

\usepackage{hyperref}
\newcommand{\myurl}[1]{\href{#1}{{\tt #1}}}%{{\tt #1}} % 

% No page numbers.
\pagestyle{empty}

\newcommand{\instructions}[1]{}
% Before submitting, comment out this line.
%\renewcommand{\instructions}[1]{{\scriptsize \sc #1}}

% The title has an 81-character maximum.
\title{Fast and flexible Bayesian platform for pharmacometric analysis}

\author{MY NAME\\
MY AFFILIATION}

\begin{document}

\maketitle

\begin{center}
An NIH STTR (R41) proposal, submitted in response to \\
{\em PAR-13-157: Early Stage Development of Technologies in Biomedical Computing, Informatics, and Big Data Science}.
\end{center}

%%%%%%%%%%%%%%%%%%%%%%%%%%%%%%%%%%%%%%%%%%%%%%%%%%%%%%%%%%%%%%%%%%%%%%%%%%%%%%%
\clearpage

\noindent
INSERT DATE HERE\\[2ex]

\noindent
Dear NIH:\\[1ex]

I am submitting a grant proposal entitled ``Fast and flexible Bayesian platform for pharmacometric analysis.''

\begin{itemize}
\item Please assign this application to the National Institute for
  General Medical Sciences
\item Please assign this application to the following study section
  Modeling and Analysis of Biological Systems Study Section [MABS] or
  the SBIR/STTR study section Small Business: Computational, Modeling, and Biodata Management [IMST (14)].
\end{itemize}
I have discussed this application with INSERT NAME OF PROGRAM OFFICER
at INSERT NAME OF INSTITUTE.  They recommended the above two
assignments.

Thank you very much.\\[1ex]
 
\noindent
Sincerely,\\[2ex]

\noindent
INSERT NAME\\
INSERT AFFILIATION


%%%%%%%%%%%%%%%%%%%%%%%%%%%%%%%%%%%%%%%%%%%%%%%%%%%%%%%%%%%%%%%%%%%%%%%%%%%%%%%
\clearpage
\section*{Project Summary/Abstract}

\instructions{ The Project Summary must contain a summary of the
    proposed activity suitable for dissemination to the public. It
    should be a self-contained description of the project and should
    contain a statement of objectives and methods to be employed. It
    should be informative to other persons working in the same or
    related fields and insofar as possible understandable to a
    scientifically or technically literate lay reader.  The Project
    Summary is meant to serve as a succinct and accurate description
    of the proposed work when separated from the application. State
    the application's broad, long-term objectives and specific aims,
    making reference to the health relatedness of the project (i.e.,
    relevance to the mission of the agency). Describe concisely the
    research design and methods for achieving the stated goals. This
    section should be informative to other persons working in the same
    or related fields and insofar as possible understandable to a
    scientifically or technically literate reader. Avoid describing
    past accomplishments and the use of the first person. Finally,
    please make every effort to be succinct. This section must be no
    longer than 30 lines of text.}


%%%%%%%%%%%%%%%%%%%%%%%%%%%%%%%%%%%%%%%%%%%%%%%%%%%%%%%%%%%%%%%%%%%%%%%%%%%%%%%
\clearpage
\section*{Project narrative}

\instructions{ Using no more than two or three sentences, describe the
    relevance of this research to public health. In this section, be
    succinct and use plain language that can be understood by a
    general, lay audience.}

%%%%%%%%%%%%%%%%%%%%%%%%%%%%%%%%%%%%%%%%%%%%%%%%%%%%%%%%%%%%%%%%%%%%%%%%%%%%%%%
\clearpage
\section*{Specific aims}

\instructions{State concisely the goals of the proposed research and
    summarize the expected outcome(s), including the impact that the
    results of the proposed research will exert on the research
    field(s) involved.  List succinctly the specific objectives of the
    research proposed, e.g., to test a stated hypothesis, create a
    novel design, solve a specific problem, challenge an existing
    paradigm or clinical practice, address a critical barrier to
    progress in the field, or develop new technology.}

The overall objective of this proposal is to develop additional
functionality within Stan to support pharmacometric applications. This
includes the addition of functions for implementing compartmental PK/PD
models and schedules of dosing and other discrete events. The
resulting enhancements to Stan will also support other applications
that would benefit from Bayesian data analysis using complex models
requiring numerical solution of systems of ordinary differential or
algebraic equations.

\subsection*{Technical Objectives}

The combined scope of the proposed Phase I and Phase II work will
result in a pharmacometric modeling and simulation platform that is
more flexible and that provides order(s) of magnitude more scalable
and more efficient Bayesian computation than available tools. This
will be accomplished by extending the capabilities of Stan to include
support for pharmacometric models. The goal is a software tool that
combines 

\begin{itemize}
\item numerical solution of systems of ordinary differential equations
  (ODEs),
\item numerical solution of systems of algebraic equations,
\item Bayesian methodology that permits integration of new data with
  prior information,
\item efficient computation of posterior densities and expectations
  with respect to them (e.g., for event probability estimation and
  prediction),
\item a flexible model specification language that permits
\begin{itemize}
\item probabilistic models that allow multiple levels and sources of
  variability and uncertainty,
\item implementation of complex models that combine submodels with
  different stochastic hierarchies. E.g., translational models that
  combine linked submodels for simultaneous analysis of clinical,
  animal and in vitro experiments,
\item model calculations that depend on a
  potentially complicated sequence of events that affect the model
  predictions (e.g., multiple dosing, changes in non-drug treatments,
  time-varying model parameters).
\end{itemize}
\end{itemize}


\subsection*{Specific Aims for Phase I}

\begin{itemize}
\item Detailed planning and feasibility assessment for development of
  a suite of Stan enhancements to support pharmacometric modeling and
  simulation
\item Development of Stan components for non-steady-state calculations
  for compartmental PK/PD models
\item Handling of discrete events, e.g., bolus or constant rate input,
  piecewise constant parameters,
\item Implementation of functions for analytic solutions for standard
  linear one and two compartment PK models with or without first order
  absorption.
\item Functions that integrate numerical solution of ODEs with the
  event handler
\item Non-stiff ODEs solved with Adams-Moulton or Runge-Kutta methods
\item Stiff ODEs solved with Backward Differentiation Formulas (BDF)
\item R package for PK/PD data handling and model implementation.
\item Linear ODEs solved semi-analytically using matrix exponentials
\item Function that integrates numerical solution of linear ODEs
  with the event handler
\end{itemize}


%%%%%%%%%%%%%%%%%%%%%%%%%%%%%%%%%%%%%%%%%%%%%%%%%%%%%%%%%%%%%%%%%%%%%
\clearpage
\section*{Research strategy}

\instructions{ Organize the Research Strategy in the specified order
    and using the instructions provided below. Start each section with
    the appropriate section heading---Significance, Innovation,
    Approach. Cite published experimental details in the Research
    Strategy section and provide the full reference in the
    Bibliography and References cited section.}

%%%%%%%%%%%%%%%%%%%%%%%%%%%%%%%%%%%%%%%%%%%%%%%%%%%%%%%%%%%%%%%%%%%%%
\subsection*{Significance}

\instructions{ Instructions: Explain the importance of the problem or
    critical barrier to progress in the field that the proposed
    project addresses.  Explain how the proposed project will improve
    scientific knowledge, technical capability, and/or clinical
    practice in one or more broad fields.  Describe how the concepts,
    methods, technologies, treatments, services, or preventative
    interventions that drive this field will be changed if the
    proposed aims are achieved.}

\instructions{ Review criteria: Does the project address an important
    problem or a critical barrier to progress in the field? If the
    aims of the project are achieved, how will scientific knowledge,
    technical capability, and/or clinical practice be improved? How
    will successful completion of the aims change the concepts,
    methods, technologies, treatments, services, or preventative
    interventions that drive this field?}

Nonlinear ordinary differential equations (ODEs) are widely used for modeling
the dynamics of complex systems in the physical, biological, and
social sciences, as well as across the engineering disciplines. Given
(noisy) measurements of system inputs and outputs, researchers are
faced with the ``inverse problem'' of estimating parameters from data
relative to a statistical model of the phenomenon at hand and the
measurement process.  Examples of application areas include
drug-disease response in pharmacology/toxicology
\citep{peterson-rigg:2010, gelman-et-al:1996}, thermal conductivity and
heat transfer for rocketry, airflow analysis over control surfaces in
in aeronautical engineering \citep{alifanov:2012}, facial recognition
in computer vision \citep{aubert-kornprobst:2006, bovik:2010}, dynamics
of gaseous bodies with respect to background magnetic fields in
astrophysics \citep{tobias-et-al:2011}, population diffusion and
spatial dynamics in ecology \citep{gopalsamy:2013}, continuous-time
asset pricing in econometrics \citep{johannes-polson:2010}, soil-carbon
respiration in biogeochemistry \citep{manzoni-porporato:2009}, cellular
regulation in systems biology \citep{baron-gastonguay:2015,
  baron-et-al:2013, leclerc-et-al:2016}, fatigue in land and sea
combat \citep{rubio-campillo:2016}, and pressure changes and oceanic
flow in fluid dynamics for meteorology \citep{charney-phillips:1953},
to name but a few.

Inference for ODEs can be challenging, both from a statistical
perspective (the solution space of a nonlinear inverse problem can
have a complex geometric structure, far from the Gaussian
distributions used in standard asymptotic inference) and also
computationally: solving the inverse problem in a probabilistic sense
requires tracing out the space of parameter values that are consistent
with data and prior information.  Even the simplest ODEs generally do
not admit closed-form solutions, hence computational approaches are
necessarily iterative, requiring a search through parameter space.
Such a search is most effectively performed using gradients, which in
turn raises the challenging problem of computing gradients of
differential equation solutions. In this setting, problems of
parameter estimation, prediction, uncertainty quantification, and
hence decision-making under uncertainty, are intractable analytically
and notoriously difficult to solve numerically.

Stan \citep{carpenter-et-al:2016, stan-development-team:2016,
  mcelreath:2016} is a widely used, open-source, probabilistic
programming language and Bayesian inference engine.  Stan currently
provides state-of-the-art Bayesian inference algorithms for inverse
problems based on gradients.  Full Bayesian inference is performed
exactly (up to some specified precision) using Hamiltonian Monte Carlo
(HMC, \citep{neal:2011}), a Markov chain Monte Carlo (MCMC) method for
sampling from the posterior and computing expectations; approximate
algorithms, which can provide exploratory analysis and even useful
estimates at orders of magitude greater throughput, include mean-field
and full-rank variational inference (VI,
\citep{wainwright-jordan:2008}) and simple optimization-based Laplace
approximation \citep{gelman-et-al:2013}.  Stan includes a simple
Runge-Kutta initial-value solver for ordinary differential equations.
% , but the solver is not optimized and does not support stiff
% differential equations, and the interface provided by Stan is at a
% very low level of core vector operations.

We propose to extend Stan to incorporate state-of-the-art solvers for
ordinary differential equations and differential algebraic equations
into its inference procedures.  Based on the performance of our
non-stiff solvers, we believe we can achieve an order of magnitude or
more speedup over the existing state-of-the-art, due to Stan's
built-in efficient automatic differentiation library and efficient
samplers for full Bayesian inference and optimizers for approximation
Bayes through variational inference and maximum marginal likelihood.
Stan provides the means to create coupled systems of differential
equations through nested automatic differentiation, which provide full
sensitivities (derivatives of solutions with respect to parameters)
\citep{lee-hovland:2002,serban-hindmarsh:2003,carpenter-et-al:2015} and
thus allow Stan's derivative-based inference engines to be used with
statistical models involving non-linear differential equations.  The
key for our proposed work will be calculating the sparse, structured
Jacobians of these coupled systems to permit solutions of stiff
systems of equations (typically characterized by varying time scales
among the components, such as the varying absorption, distribution,
metabolism, and excretion of toxins in bone, fat, blood, and organs
such as kidneys).

We will also extend Stan to deal with events arising from external
inputs such as discrete (bolus) dosing in pharmacology or leaf-litter
decomposition by enzymes in soil-carbon modeling. These represent
general improvements in the Stan language which should serve specific
customers in the present commercial development.  In that way, we
propose at Columbia University to serve the general science and
engineering community with improvements in open-source software while
at the same time at Metrum producing a product and services for which
there is a strong market.

For concreteness, we will evaluate the tools produced using
pharmacometric data with a range of sophisticated statistical and
mathematical models in common use \citep[e.g.,][]{ette-williams:2007,
  schmidt-derendorf:2014}. Our goal is to extend the size of data
sets, sophistication of models (e.g., varying effects by patient,
missing data, meta-analysis of multiple drug studies and placebo
controls, increased granularity of spatio-temporal modeling,
hierarchical modeling of (sub)populations), and speed of solvers by an
order of magnitude or more over the state of the art, a target we
believe is realistic given Stan's performance with simple non-stiff
ODE solvers \citep{weber-et-al:2014}.  The result will be an even more
flexible Bayesian statistics platform that supports analysis of
heterogeneous collections of data conditioned on models of great
stochastic and deterministic complexity and quantitative prior
knowledge, leading to better calibrated and sharper (more precise)
predictions \citep[see, e.g.,][]{gneiting-et-al:2007}.

\subsection*{Background}

Modern research in science and engineering is characterized by big,
messy data. Scientific datasets are not just bigger in that there are
more instances of the same thing---a mere increase in sample size
would make our job as data analysts easier. Rather, there is more {\em
  breadth} and {\em complexity} to the data: more subgroups,
locations, or time granularity than is currently being modeled, more
partial and noisy measurements that cannot easily be incorporated into
standard models, more related studies available for meta-analysis and
prior formulation, more information on the physical mechanisms of
measurement, more information on the population units being measured,
and more fine-grained information on the predictions desired.

Consequently, in order to utilize big data we need complementary big
models that can, for example, adjust the data to match sample to
population, to match treatment to control groups in causal inference,
or learn about interactions and variation and make individualized
predictions rather than be limited to averages.  Once we have big
models we inevitably have multiple levels of uncertainty and
variation, which can be studied using Bayesian inference, as is
demonstrated in a large literature of applied statistics, including
our own books and research articles. The big computational challenge
is scalability: developing algorithms and implementations that work
well and do not require too much storage and computation time as
datasets become larger and the amount of predictive information
increases.

\subsection*{Modeling and simulation in drug research and development}

Pharmacometric data typically consists of longitudinal measurements
gathered from different sources such as drug concentrations in blood
plasma and pharmacological effects often resulting from a sequence of
treatment events, e.g., drug doses. Mathematical modeling and
simulation facilitates decision-making and risk-benefit assessment by
drug developers regulators, health care providers, and patients.

Mathematical models describing measurements over time are often
compartmental models formulated as systems of first order ordinary
differential equations (ODEs). These models often also involve a
hierarchy of random effects required to describe sources of
variability, e.g., inter-individual, inter-occasion and residual
variation. Highly complex systems pharmacology models describe
physiologic, biochemical, and pharmacological processes at multiple
scales (molecular, cellular, tissue, organ, organism and population)
\citep[e.g.,][]{baron-et-al:2013, peterson-rigg:2010}. Even relatively
simple compartmental models used for pharmacokinetic data analysis
require solution of systems of ODEs often by computationally demanding
numerical methods.

Although there are well-developed methods for solving continuous ODEs
given initial values, tools for inference in models that incorporate
ODEs as part of the statistical model are less-developed. Discrete
events such as bolus doses into a model compartment induce
discontinuities in the ODEs or their solution, which adds additional
computational complexity. We propose to evaluate and extend software
tools to incorporate better methods for solving ODEs and handle
discrete events.

Systems pharmacology models are very useful for applications in
translational drug research--predictions of drug disposition and
effects in humans based on preclinical research. The development of
such models is highly challenging and resource intensive. It requires
integration of large amounts of prior biochemical and physiological
knowledge with a heterogeneous collection of data from in vitro,
animal and human studies. Bayesian methods provide a rational approach
to the model-based analysis of data conditioned on such complex models
and quantitative problem-specific knowledge about the model parameters
\citep[e.g.,][]{gelman-et-al:1996, leclerc-et-al:2016}.

%%%%%%%%%%%%%%%%%%%%%%%%%%%%%%%%%%%%%%%%%%%%%%%%%%%%%%%%%%%%%%%%%%%%%
\subsection*{Innovation}

\instructions{ Instructions: Explain how the application challenges and
    seeks to shift current research or clinical practice paradigms.
    Describe any novel theoretical concepts, approaches or
    methodologies, instrumentation or intervention(s) to be developed
    or used, and any advantage over existing methodologies,
    instrumentation or intervention(s).  Explain any refinements,
    improvements, or new applications of theoretical concepts,
    approaches or methodologies, instrumentation or interventions.}

\instructions{ Review criteria: Does the application challenge and seek
    to shift current research or clinical practice paradigms by
    utilizing novel theoretical concepts, approaches or methodologies,
    instrumentation, or interventions? Are the concepts, approaches or
    methodologies, instrumentation, or interventions novel to one
    field of research or novel in a broad sense? Is a refinement,
    improvement, or new application of theoretical concepts,
    approaches or methodologies, instrumentation, or interventions
    proposed? }

\subsection*{Comparison to Existing Tools}\label{existing}

Various existing software tools address subsets of those features, but
none address them all. Shortcomings of currently available software
are outlined below. Commercial tools that specialize in pharmacometric
applications such as NONMEM, Monolix and Phoenix offer

\begin{itemize}
\item Good support for compartmental models based on ODEs and for
  typical schedules of dosing and other discrete events.
\item Including ODE solvers specialized for stiff, non-stiff and
  linear ODEs
\item Limited flexibility w.r.t. probabilistic model components.
\item NONMEM \& Monolix employ relatively inefficient MCMC algorithms
  for Bayesian analyses.
\item Phoenix provides no support for Bayesian analysis.
\item Stochastic structure of supported models is limited:
\item Only Normal-Wishart priors
\item Only normally distributed variability (except for residual
  variability)
\item Cannot combine submodels with different stochastic hierarchies
\end{itemize}

Stan is in part modeled on the previous state-of-the-art tool for
Bayesian inference via MCMC, WinBUGS \citep{lunn-et-al:2000}, which
contains a module PKBugs for ODE solving, and for which we have the
BUGSModelLibrary of sample models for testing
\citep{gastonguay-et-al:2010}.  WinBUGS and other Gibbs-sampling based
approaches provide

\begin{itemize}
\item inefficient MCMC algorithms compared to Stan, which is
\begin{itemize}
\item  two orders of magnitude more scalable, and
\item  between two times faster and orders of magnitude faster,
  with greater speedups in more challenging problems.
\end{itemize}
\end{itemize}

\subsection*{Advantages of Stan}

In contrast, Stan provides

\begin{itemize}
\item more efficient posterior simulation algorithms (HMC/NUTS) than
  other pharmacometrics and general purpose Bayesian analysis tools,
\item approximate Bayesian (ADVI) and optimization methods,
\item numerical solvers for non-stiff ODEs,
\item only limited, primitives-based support for pharmacometrics models,
\item no tools for handling typical schedules of dosing and other discrete events,
\item a single simple ODE solver (no stiff solvers, no efficient
  solvers for linear ODEs), and
\item no tools for solving nonlinear algebraic equations, which are
  required for steady-state solutions for ODES, a commonly encountered
  challenge in pharmacometric analyses.
\end{itemize}

These properties of Stan make it a highly attractive general purpose
program for Bayesian data analysis and arguably the best available
platform upon which to build additional features. As outlined above,
Stan lacks some important capabilities required for many
pharmacometric applications.

%%%%%%%%%%%%%%%%%%%%%%%%%%%%%%%%%%%%%%%%%%%%%%%%%%%%%%%%%%%%%%%%%%%%%
\subsection*{Approach}

\instructions{ Instructions: Describe the overall strategy,
    methodology, and analyses to be used to accomplish the specific
    aims of the project. Unless addressed separately in Item 5.5.15,
    include how the data will be collected, analyzed, and interpreted
    as well as any resource sharing plans as appropriate.  Discuss
    potential problems, alternative strategies, and benchmarks for
    success anticipated to achieve the aims.  If the project is in the
    early stages of development, describe any strategy to establish
    feasibility, and address the management of any high risk aspects
    of the proposed work.}

\instructions {For new applications, include information on
  Preliminary Studies as part of the Approach section. Discuss the
  PD/PI'ェs preliminary studies, data, and/or experience pertinent to
  this application. Except for Exploratory/Development Grants
  (R21/R33), Small Research Grants (R03), Academic Research
  Enhancement Award (AREA) Grants (R15), and Phase I Small Business
  Research Grants (R41/R43), preliminary data can be an essential part
  of a research grant application and help to establish the likelihood
  of success of the proposed project.}

\instructions{ Review criteria: Are the overall strategy, methodology,
    and analyses well-reasoned and appropriate to accomplish the
    specific aims of the project? Are potential problems, alternative
    strategies, and benchmarks for success presented? If the project
    is in the early stages of development, will the strategy establish
    feasibility and will particularly risky aspects be managed?}


\subsection{Phase I Base}

Phase I Base will be limited to (1) detailed planning and feasibility
assessment for development of a suite of Stan enhancements to support
pharmacometric modeling and simulation, and (2) developing an initial
subset of the planned components.

\subsubsection{Planning and feasibility assessment}

The first stage of the work will be to develop more detailed
functional specifications for Stan components that fulfill the
following general functional requirements: Functions that calculate
the amount in each compartment of a compartmental PK/PD model at a
specified time given the amount in those compartments at a previous
time, Calculations based on analytic solutions for the standard linear
one and two compartment PK models with or without first order
absorption. Calculations based on numerical solution of a
user-specified system of ordinary differential equations (ODEs).
Specific functions will be required for different categories of ODEs
\citep{byrne-hindmarsh:1975}:

\begin{itemize}
\item Stiff ODEs solved numerically via methods based on backward
  differentiation formulas (BDF).
\item Non-stiff differential equations solved numerically via
  Adams-Moulton or Runge-Kutta methods.
\item Linear ODEs solved semi-analytically via matrix exponential methods.
\item Functions that calculate the amount in each compartment of a
  compartmental PK/PD model at a vector of specified times given a
  schedule of discrete events, e.g., bolus or constant rate doses and
  piecewise constant parameters. These functions will be developed by
  integrating the functions for calculating the compartment amounts at
  a single time with software for handling the event schedule. 
\item One or more functions for numerical calculation of the solution
  to a system of nonlinear algebraic equations.
\item Functions that calculate the amount in each compartment of a
  compartmental PK/PD model at a vector of specified times under
  quasi-steady-state conditions resulting from a periodic input, e.g.,
  repeated administration of equal doses at equally spaced time
  intervals. This requires numerical solution of both ODEs and
  algebraic equations in order to solve a boundary value problem.
\item Integral with the development of the above functional
  specifications will be assessing the feasibility of implementing
  them within Stan. Feasibility assessment may also extend to
  consideration of additional functionality that may be desirable for
  pharmacometric and other scientific modeling applications, e.g.,
  solutions of delay differential equations, stochastic differential
  equations, differential algebraic equations and partial differential
  equations.
\end{itemize}

\subsubsection{Development of Stan components for non-steady-state
  calculations for compartmental PK/PD models} 

The software developed in this project will be written in C++. The C++
code will result in new functions in the Stan model specification
language. Subject to revision based on the planning and feasibility
effort described above, C++ code consistent with Stan architecture
will be developed for the following tasks. A function will be
constructed that calculates the solution of a system of ODEs at a
specified sequence of times given a schedule of discrete events, e.g.,
bolus or constant rate input or piecewise constant parameters. This
will be a general event handler that calls another function that
returns the solution to a specific ODE system at a single time given
initial values, i.e., implementing a recursive approach in which the
ODEs are solved over each continuous interval given initial conditions
for that interval calculated from the solution for the prior
interval. Functions for calculating analytic solutions for standard
linear one and two compartment PK models with or without first order
absorption will be created.  A function for calculating numerical
solutions of non-stiff ODEs using an Adams-Moulton method will also be
implemented. This will probably employ code from the Sundials
collection \citep{hindmarsh-et-al:2005}. Stan functions have already
been developed for calculating numerical solutions of ODEs using
Backward Differentiation Formulas (BDF) for stiff ODEs and Runge-Kutta
methods for non-stiff ODEs.

Additional C++ code will be written that integrates the ODE solution
functions with the event handler and exposes the results as functions
in the Stan language. In addition a general purpose ODE solver
function using the Adams-Moulton method will be added to Stan. The function
will be consistent with the current Stan implementations of the
Runge-Kutta and BDF solvers.

\subsubsection{Software development practices}

Stan is developed using current best-practices for production software
development.  These rely on a combination of test-driven interface
design, established coding standards, thorough code reviews,
exhaustive unit tests, and ongoing continous integration testing
on all supported platforms.

Stan's software distributions, including the source version control
repositories and the official releases, along with the Wiki used for
design documents, and the issue tracker for feature requests and bug
reports, are hosted by GitHub.%
%
\footnote{\myurl{http://github.com/}}
%
The entire distribution and the in-development code are open access,
and available through the public-facing GitHub web site.  All work is
carried out in branches and submitted to be merged with the
development branch through pull requests which are openly code
reviewed.  All behavior must be thoroughly tested, both with unit
tests for the local functionality and integration tests to guard
against breaking things end-to-end with local changes.


\subsubsection{User-facing changes}

Stan has over 1500 registered users on its users list and has seen
over 200 academic publications across science, engineering, business,
and sports.  The results of this project will involve extensive
additions to the user-facing documentation. This includes
interface-by-interface walk-through tutorials and example models. Stan
is increasingly being included in third-party textbooks, and we strive
for backward compatibility so that these references are not made
obsolete. Stan also has a dedicated web site, which is also managed
through GitHub.

\subsubsection{Collaboration plan}

The project will be a close collaboration between personnel at Metrum
Research Group and the Stan development team based at Columbia
University. Metrum scientists provide expertise in pharmacometrics. In
most cases they will be responsible for initial functional
specifications and will often develop prototype code for the desired
functionality in Stan. The Stan team will collaborate with Metrum
scientists to refine the functional specifications. They will adapt or
replace the prototype code to develop Stan code suitable for inclusion
in publicly released versions of Stan. Part of that process will
include rigorous testing to assure the code functions as intended.

The main coordination mechanisms we use are GitHub for version control
and code review, Jenkins for continuous integration testing, weekly
Google+ hangouts for online meetings as well as additional meetings as
needed.  In this particular case, we are within public transportation
distance of each other, which will facilitate regular face-to-face
meetings with whiteboards.


\subsection{Phase I Option}

\subsubsection{R package development}

R (\url{https://www.r-project.org/}) is a statistics software package widely used by
pharmacometricians and other scientific fields. Users can develop
additional R components and make them available as easily installed
packages. An R package will be developed to provide a more
familiar and user-friendly tool for the following tasks:

\begin{itemize}
\item Translation of data sets formatted for NONMEM
  (\url{http://www.iconplc.com/innovation/solutions/nonmem/}) into
  Stan-compatible formats
\item Model specification
\item Model execution using either the rstan or cmdstan interfaces to
  Stan
\item Tabular and graphical analyses of Stan output specialized for
  pharmacometric applications
\item The component functions will be programmed in the R language.
  They will be made available to the community via open distribution
  sites, e.g., CRAN (\url{https://cran.r-project.org/}) or R-Forge
  (\url{https://r-forge.r-project.org/}).
\end{itemize}

\subsubsection{Development of additional Stan components for linear compartmental PK/PD models}

A large fraction of pharmacometric models are based on systems of
linear ODEs with constant coefficients. Such models can be solved
semi-analytically via a matrix exponential approach. A matrix
exponential method can solve applicable ODEs much more rapidly than
the multi-step numerical methods implemented as part of Phase I Base.
A Stan function for calculating non-steady-state and steady-state
solutions of linear ODEs using matrix exponentials will be developed.
The specific computational approach will be identified during the
Phase I planning. A variety of algorithms have been proposed
\citep{moler-vanloan:2003} and implemented in various languages, e.g.,
FORTRAN \citep{sidje:1998} and C++
(\url{https://people.sc.fsu.edu/~jburkardt/cpp_src/matrix_exponential/matrix_exponential.html}).

As with the ODE solvers described in section 3.1.2, a Stan function
integrating the matrix exponential solver with the event handler will
be developed.

%%%%%%%%%%%%%%%%%%%%%%%%%%%%%%%%%%%%%%%%%%%%%%%%%%%%%%%%%%%%%%%%%%%%%%%%%
\clearpage

\instructions{Provide a bibliography of any references cited in the
  Research Plan. Each reference must include names of all authors (in
  the same sequence in which they appear in the publication), the
  article and journal title, book title, volume number, page numbers,
  and year of publication. Include only bibliographic
  citations. Follow scholarly practices in providing citations for
  source materials relied upon in preparing any section of the
  application.  The references should be limited to relevant and
  current literature. While there is not a page limitation, it is
  important to be concise and to select only those literature
  references pertinent to the proposed research.}

\instructions{When citing articles that fall under the Public Access
  Policy, were authored or co- authored by the applicant and arose
  from NIH support, provide the NIH Manuscript Submission reference
  number (e.g., NIHMS97531) or the PubMed Central (PMC) reference
  number (e.g., PMCID234567) for each article. If the PMCID is not yet
  available because the Journal submits articles directly to PMC on
  behalf of their authors, indicate ``PMC Journal - In Process.''
  Citations that are not covered by the Public Access Policy, but are
  publicly available in a free, online format may include URLs or
  PMCID numbers along with the full reference (note that copies of
  these publications are not accepted as appendix material, see
  Section 5.7).}

% Rename the bibliography section.
\bibliographystyle{nihunsrtnat}
%\bibliographystyle{plainnat}
\renewcommand{\refname}{Literature Cited}
\bibliography{nih}

%%%%%%%%%%%%%%%%%%%%%%%%%%%%%%%%%%%%%%%%%%%%%%%%%%%%%%%%%%%%%%%%%%%%%%%%%
\clearpage
\section*{Consortium/contractual arrangements}

%%%%%%%%%%%%%%%%%%%%%%%%%%%%%%%%%%%%%%%%%%%%%%%%%%%%%%%%%%%%%%%%%%%%%%%%%
\clearpage
\section*{Letters of support}

%%%%%%%%%%%%%%%%%%%%%%%%%%%%%%%%%%%%%%%%%%%%%%%%%%%%%%%%%%%%%%%%%%%%%%%%%
\clearpage
\section*{Resource sharing plan(s)}

\subsection*{Data sharing plan}

%%%%%%%%%%%%%%%%%%%%%%%%%%%%%%%%%%%%%%%%%%%%%%%%%%%%%%%%%%%%%%%%%%%%%%%%%
\clearpage
\section*{Budget justification}

\paragraph{Personnel}

\begin{itemize}

\item 
\end{itemize}

%%%%%%%%%%%%%%%%%%%%%%%%%%%%%%%%%%%%%%%%%%%%%%%%%%%%%%%%%%%%%%%%%%%%%%%%%
\clearpage
\section*{Facilities \& Other Resources}

\instructions{ Instructions: This information is used to assess the
    capability of the organizational resources available to perform
    the effort proposed.  Identify the facilities to be used
    (laboratory, clinical, animal, computer, office, other). If
    appropriate, indicate their capacities, pertinent capabilities,
    relative proximity and extent of availability to the
    project. Describe only those resources that are directly
    applicable to the proposed work. Provide any information
    describing the Other Resources available to the project (e.g.,
    machine shop, electronic shop) and the extent to which they would
    be available to the project.  Describe how the scientific
    environment in which the research will be done contributes to the
    probability of success (e.g., institutional support, physical
    resources, and intellectual rapport). In describing the scientific
    environment in which the work will be done, discuss ways in which
    the proposed studies will benefit from unique features of the
    scientific environment or subject populations or will employ
    useful collaborative arrangements.}

\instructions{ Review criteria: Will the scientific environment in
    which the work will be done contribute to the probability of
    success? Are the institutional support, equipment and other
    physical resources available to the investigators adequate for the
    project proposed? Will the project benefit from unique features of
    the scientific environment, subject populations, or collaborative
    arrangements?}

\paragraph{Office}

\paragraph{Computer}

\paragraph{Other}

%%%%%%%%%%%%%%%%%%%%%%%%%%%%%%%%%%%%%%%%%%%%%%%%%%%%%%%%%%%%%%%%%%%%%%%%%
\clearpage
\section*{Equipment}

\subsection*{Metrum Research Group}

Since the primary focus of this project is computer software and
development, the equipment requirements are primarily suitable
computer hardware and software resources. All Metrum scientists are
equipped with a recent model Apple MacBook Pro. Most production
computation is done on Amazon Web Services (AWS) cloud compute servers
using the Ubuntu operating system. The software required for the
proposed project is installed on the standard machine image
configuration used by Metrum, e.g., Gnu C++ compiler, R, Gnu Fortran,
NONMEM, LaTeX, Stan, git and subversion version control software.

\subsection*{Columbia University}

At Columbia University, all participants are equipped with recent
Apple Macbook Pro personal computers.  The Stan project has three
dedicated servers with 64GB and 8 cores each, one for each of Mac OS
X, Linux, and Windows, along with two dedicated workstations for web
hosting and remote debugging on Windows and Linux.  The Stan team has
access to two Linux clusters, one shared by statistics and astronomy
with 384 cores, and one shared by 24 departments with 2672 CPU cores,
35 of which are high end 256-gigabyte memory nodes, and 45,000+ NVIDIA
K40 and K20 GPU cores.  All of the software required for individual
computers for this project (see the list above) is free and open
source.  Source version control, issue trackers, and project web
servers are provided by GitHub, mailing lists for user and developer
groups is supplied by Google, and continuous integration for the
interfaces by Travis;  all of these services are free to open-source
projects.


\end{document}
