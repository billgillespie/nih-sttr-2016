% -*-LATEX-*-
% Document name: 
% Creator: Rob MacLeod [macleod@cvrti.utah.edu]
% Creation Date: 
%%%%%%%%%%%%%%%%%%%%%%%%%%%%%%%%%%%%%%%%%%%%%%%%%%%%%%%%%%%%%%%%%%%%%%
\documentclass[10pt]{article}
\usepackage{times}
\usepackage[super]{cite}
\usepackage{fancyhdr}
\usepackage{graphicx}
\usepackage{bibunits}
\usepackage{hyperref}
\renewcommand\rmdefault{phv}
\textheight 9.6in
\textwidth 7.2in
\topmargin -.95in
\oddsidemargin -.45in
\pagestyle{fancy}
\headsep .4in
\footskip 25pt
\headheight 15pt
\setlength{\parskip}{\smallskipamount}
% Pick the first one for the final submission, the second for drafts
\pagestyle{empty}
%\pagestyle{plain}

\newcommand{\eg}{{\em e.g.,}}
\newcommand{\ie}{{\em i.e.,}}
\newcommand{\etc}{{\em etc.,}}
\newcommand{\etal}{{\em et al.}}
\newcommand{\degrees}{{$^{\circ}$}}
\newcommand{\splitline}{\begin{center}\rule{\columnwidth}{.7mm}\end
{center}}

% Define some variables for this particular edition
\newcommand{\myname}{Robert S. MacLeod}
\newcommand{\eraname}{rsmacleod}
\newcommand{\mytitle}{Professor of Bioengineering }
%\setcounter{page}{1}
%\newcommand{\totalpubs}{110}

% Header content
% \renewcommand{\headrulewidth}{0pt}
% \lhead{\hspace{.75in}{\scriptsize\sf Pricipal Investigator/Program 
%     Director(Last, First, Middle): }
% \hspace{.1in}\PI}
% %\rhead{Core \thechapter}
% %\lfoot{{\small\sf  PHS 398/2590 (Rev.~09/04)}}
% \cfoot{{\scriptsize Page: \underline{\hspace{1em}\thepage\hspace{1em}}}}
% %\rfoot{{\small Continuation Format}}

\def\thebibliography#1{%\subsection*{C. Selected peer-reviewed publications}
\list
 {\arabic{enumi}.}{\settowidth\labelwidth{[#1]}\leftmargin\labelwidth
 \advance\leftmargin\labelsep
 \usecounter{enumi}}
% \def\newblock{\hskip .11em plus .33em minus .07em}%
 \def\newblock{\hskip .07em}
 \sloppy\clubpenalty4000\widowpenalty4000
 \sfcode`\.=5000\relax}
\let\endthebibliography=\endlist

%%%%%%%%%%%%%%%%%%%%%%%%%%%%%%%%%%%%%%%%%%%%%%%%%%%%%%%%%%%%%%%%%%%%%%

\begin{document}
%\pagestyle{fancy}
\pagestyle{empty}


%\setlength\extrarowheight{40pt}
\renewcommand{\arraystretch}{1.4}
\noindent
\begin{tabular}[h]{p{2.8in}|c|p{0.75in}|l} \hline 
%    \multicolumn{4}{c}{}\\
    \multicolumn{4}{c}{\textbf{BIOGRAPHICAL SKETCH}}\\
    \multicolumn{4}{p{\columnwidth}}
    {\centering{\footnotesize Provide the following information for the
  Senior/key personnel and other significant contributors\\
  Follow this format for each person   
  \uppercase{DO NOT EXCEED FIVE PAGES}}} \\
%  \multicolumn{4}{c}{}\\ 
\hline
 \multicolumn{1}{l}{NAME:} & \multicolumn{3}{l}{\myname} \\ \hline
\multicolumn{1}{l}{eRA COMMONS USER NAME:} & 
                  \multicolumn{3}{l}{\eraname} \\ \hline
\multicolumn{1}{l}{POSITION TITLE:} & \multicolumn{3}{l}{\mytitle} \\ \hline
\multicolumn{4}{p{\columnwidth}}{ EDUCATION/TRAINING 
  (Begin with baccalaureate
  or other initial professional education, such as nursing, and include
  postdoctoral training and residency training if applicable.
  Add/delete rows as necessary.)} \\ \hline
\multicolumn{1}{c|}{ \uppercase{Institution and location}} & 
\multicolumn{1}{p{1 in}|}{\centering \uppercase{Degree} (if applicable)} & 
\multicolumn{1}{p{.75 in}|}{\centering Completion Date  MM/YYYY} & 
\multicolumn{1}{c}{\ \uppercase{Field of Study}} \\ \hline
Dalhousie University, Halifax, N.S., Canada & BSc & \centering{1979} & 
 Engineering Physics\\
Technische Universit\"at Graz, Graz, Austria & MSc & \centering{1985} & 
 Electrical Engineering \\
Dalhousie University, Halifax, N.S., Canada & Ph.D. & \centering{1990} & 
Physiology \& Biophysics \\
CVRTI, University of Utah, Salt Lake City, Utah & Postdoctoral & 
\centering{1992} & Electrophysiology \\
\hline	

\end{tabular}
\renewcommand{\arraystretch}{1.0}

%\vspace{-.2in}

\subsection*{A. Personal Statement}

I have over 30 years of experience in basic and translational cardiac
electrophysiology, biomedical simulation, image based modeling, and
visualization.  My approaches include human and animal experimental studies
using a wide range of technologies, especially multichannel measurement of
electrical signals, image analysis, image based modeling, and numerical
techniques for large scale computations.  For 13 years I have been the
co-director of an NIH funded Center for Integrated Biomedical Computing
(CIBC), in which we develop open source software for use by a range of
biomedical scientists and engineers.  One focus of my recent research is on
all aspects of atrial fibrillation detection, evaluation, and treatment and
our software has had a major impact in the clinical management of this
disease.  My current research funding is over \$600,000 per year and I
mentor 5 PhD students, a post doctoral fellow, 10 undergraduates, and
several software developers. I have extensive past experience mentoring
over 20 graduate students, post doctoral fellows, and junior faculty and am
the Vice-chair and Director of Undergraduate Studies in Bioengineering.
This proposal for Ravi Ranjan builds directly on all this research and
mentorship experience and is the ideal next step after Ravi's successful K
award and his outstanding progress in the last three years.  I am
enthusiastic about this opportunity, especially with the chance to continue
to collaborate closely with Ravi.  He is an outstanding young researcher
with unparalleled pedigree and an already outstanding track record, one
that is advancing very nicely since his arrival in Utah.  

The specific approaches in this proposal match very well my own expertise
and the well established capabilities at the University of Utah. There
exists deep expertise and long experience in both MRI imaging (at the Utah
Center for Advanced Imaging Research, UCAIR) and image analysis (at the
Scientific Computing and Imaging, SCI, Institute) and in the development
and use of high density electrocardiographic mapping techniques (at the
Cardiovascular Research and Training Institute, CVRTI).  As Co-Director of
the NIH/NIGMS funded Center for Integrative Biomedical Computing (CIBC), I
have also guided the development of software that will be extremely useful
for the proposal research, especially the image and signal analysis
components. My role as Associate Director of both SCI and CVRTI along with
the close collaborations with Dr. Ranjan create the uniquely ideal
environment for the proposed research into mechanisms of persistent AF and
improvements in ablation based management of this debilitating and costly
condition.

\noindent
% Define the bib files to use (see \nocite command, below)
\bibliography{strings,biglit}
\bibliographystyle{nihunsrt}
\nocite{RSM:Ran2014,RSM:Par2014,RSM:Ran2012,RSM:Dos2013}

\subsection*{B. Positions and Honors}
\vspace{-.1in}

\noindent
\textbf{Positions and Employment}\\
\begin{tabular}[h]{lp{6.1in}}
    1980--86 & Research Assistant Professor and Computer System Manager,
    Institut f\"ur Medizinische Physik und Biophysik, Universit\"at Graz, Graz,
    Austria \\
    1986--90 & Graduate Research Assistant, Department of Physiology \&
    Biophysics, Dalhousie University, Halifax, N.S., Canada, Postdoctoral
    Research Associate, Nora Eccles Harrison Cardiovascular Research and
    Training Institute, University of Utah School of Medicine, Salt Lake
    City, Utah, 1990-1992 \\
    1992--1998 & Research Assistant Professor, Nora Eccles Harrison
    Cardiovascular Research and Training Institute, University of Utah
    School of Medicine, Salt Lake City, Utah\\
    1993--1997 & Research Assistant Professor, Department of
    Bioengineering, University of Utah, Salt Lake City, Utah \\
    1998--2003 & Assistant Professor, Department of Bioengineering,
    University of Utah, Salt Lake City, Utah \\
    2003--2012 & Tenured Associate Professor, Department of
    Bioengineering, University of Utah, Salt Lake City, Utah \\
    1999--2003 & Acting Co-Director, Cardiovascular Research and Training
    Institute, University of Utah  \\
    2003--present & Associate Director, Cardiovascular Research and Training
    Institute, University of Utah \\
    2002--present & Associate Director, Scientific Computing and Imaging
    (SCI) Institute, University of Utah \\
    2006--present & Associate Chair and Head of Undergraduate Studies,
    Dept. of Bioengineering, U of Utah \\
    2009--present & Co-founder and Associate Director Comprehensive
    Arrhythmia Research and Management (CARMA) Center \\
    2012--present & Full  Professor, Department of
    Bioengineering, University of Utah, Salt Lake City, Utah \\
\end{tabular}

\noindent
\textbf{Other Experience and Professional Memberships}\\
\begin{tabular}[h]{lp{6.1in}}
1990-- & Member, IEEE Engineering in Medicine and Biology Society (EMBS) \\
1999-- & Member Biomedical Engineering Society (BMES)  \\
2008-- & Member, Heart Rhythm Society (HRS) \\
1999-- & International Society of Electrocardiology (ISE) \\
2002-- & ISE International Council member\\
1979-- & Computing in Cardiology (CinC) Society \\
2014-- & CinC Council member \\
1995-- & International Society of Computerized Electrocardiology\\
1995-- & Ad hoc NIH study section member \\
2005-- & Editorial Board, Journal of Electrocardiology\\
\end{tabular}

% \newpage
\noindent
\textbf{Honors}\\
\begin{tabular}[h]{ll}
    1975--79 & Dalhousie University Academic Scholarships\\
    1979     & Dalhousie University Medal in Engineering Physics \\
    1980--85 & Austrian Student Union Scholarships \\
    1986--90 & Izaak Walton Killam Memorial Scholarship \\
    1987--90 & Medical Research Council of Canada Studentship \\
    1990--92 & Heart and Stroke Foundation of Canada Postdoctoral Fellowship\\
\end{tabular}

\subsection*{C. Contribution to Science)}

\begin{description}
  \item [Electrocardiographic Inverse Problems: ] I have spent my entire
    career exploring novel approaches to inverse problems in
    electrocardiography, \ie{} the estimation of cardiac electrical
    activity from body-surface ECG.  I have pursued a wide variety of
    approaches and used both computational and experimental methods to
    develop and validate a range of approaches, and most recently have
    applied uncertainty quantification approaches to evaluate the role of
    parameter variation on the accuracy of the resulting solutions to the
    forward and inverse problems.
    
    \begin{enumerate}
      \item M. Milanic and V. Jazbinsek and R.S. MacLeod and D.H. Brooks
        and R. Hren. Assessment of regularization techniques for
        electrocardiographic imaging in \emph{J Electrocardiol}
        47(1):20--28, 2014. PMCID: PMC4154607.

      \item D. Wang, R.M. Kirby, R.S. MacLeod, and C.R. Johnson. Inverse
        electrocardiographic source localization of ischemia: An
        optimization framework and finite element
        solution. \emph{J. Comp. Phys.}  250:403--424, 2013.  PMCID:
        PMC3727301

      \item D.J. Swenson, S.E. Geneser, J.G. Stinstra, R.M. Kirby, and
        R.S. MacLeod. Cardiac Position Sensitivity Study in the
        Electrocardiographic Forward Problem Using Stochastic Collocation
        and BEM. \emph{Ann. Biomed. Eng.}, 39(12):2900--2910, 2011. PMCID:
        PMC3362042.

      \item Y. Serinagaoglu, D.H. Brooks, and R.S. MacLeod. Improved
        performance of Bayesian solutions for inverse electrocardiography
        using multiple information sources. \emph{ IEEE
          Trans. Biomed. Eng.} 53(2):2024--2034, 2006.
    \end{enumerate}

  \item [Image based biomedical modeling:] This research direction stemmed
    from the recognition of the absence of tools for conducting patient
    specific, imaged based modeling and simulation of physiology.  To
    extend and make openly available the tools we have developed for the
    purpose, we have received NIH support since 1999 and now release a
    complete pipeline of software for image based modeling and simulation.
    We conduct courses each year (some under separate NIH funding) and have
    identified over 300 publications from researchers who use our software
    and techniques for their research.

    \begin{enumerate}
      \item J. Blauer, D.J. Swenson, K. Higuchi, G. Plank, R. Ranjan,
        N. Marrouche, and R.S. MacLeod. Sensitivity and Specificity of
        Substrate Mapping: an in silico Framework for the Evaluation of
        Electroanatomical Substrate Mapping Strategies in
        \emph{J. Cardiovasc. Electrophys.} 25(7): 774--780, 2014.  PMCID:
        PMC4107007. 
      \item B.M. Isaacson, J.G. Stinstra, R.D. Bloebaum, COL P.F. Pasquina,
        and R.S. MacLeod. Establishing Multiscale Models for Simulating
        Whole Limb Estimates of Electric Fields for Osseointegrated
        Implants.  \emph{IEEE Trans. Biomed. Eng.} Oct;58(10):2991--4,
        2011.  PMCID: PMC3179554. 
      \item R.S. MacLeod, J.G. Stinstra, S. Lew, R.T.  Whitaker,
        D.J. Swenson, M.J. Cole, J. Kruger, D.H. Brooks, and
        C.R. Johnson. "Subject-specific, multiscale simulation of
        electrophysiology: a software pipeline for image-based models and
        application examples.  \emph{Phil. Trans.  Royal Soc.}
        367(1896):2293--2310, 2009. PMCID: PMC2696107.
      \item S.E. Geneser, R.M. Kirby, and R.S. MacLeod. Application of
        Stochastic Finite Element Methods to Study the Sensitivity of ECG
        Forward Modeling to Organ Conductivity, In \emph{IEEE Transactions
          on Biomedical Engineering}, 55(1):31--40, 2008.  
    \end{enumerate}

  \item [Simulation of cardiac arrhythmias and defibrillation: ] We have
    conducted, in collaboration with physicians and other simulation
    researchers, studies based on simulation of both cardiac arrhythmias
    and the therapeutic management of their extreme forms through
    defibrillation. Our arrhythmia simulations focus on identifying the
    role of abnormal tissue substrate in supporting tachycardia and
    fibrillation. The goals of our defibrillation studies have focused on
    optimizing electrode and device placement in order to minimize energy
    needs while achieving successful defibrillation. 

    \begin{enumerate}
      \item K.S. McDowell, S.S. Zahid, F. Vadakkumpadan, J.  Blauer,
        R.S. MacLeod, and N.A. Trayanova.  Virtual Electrophysiological
        Study of Atrial Fibrillation in Fibrotic Remodeling, \emph{PLoS
          ONE},10(2), 2015.  PMCID: PMC4333565.
      \item K.S. McDowell, F. Vadakkumpadan, R. Blake, J.  Blauer,
        G. Plank, R.S. Macleod, and N.A. Trayanova.  Mechanistic inquiry
        into the role of tissue remodeling in fibrotic lesions in human
        atrial fibrillation. \emph{Biophys. J.} 104(12):2764--2773, 2013.
        PMCID: PMC3686346. 
      \item M. Jolley, J. Stinstra, J. Tate, S. Pieper, R. Macleod, L. Chu,
        P. Wang, and J.K. Triedman.  Finite element modeling of
        subcutaneous implantable defibrillator electrodes in an adult
        torso.  \emph{Heart Rhythm.} 7(5):692--698, 2010. PMCID:
        PMC3103844.
      \item B. Taccardi, B.B. Punske, E. Macchi, R.S. MacLeod,
        P.R. Ershler.  Epicardial and Intramural Excitation During
        Ventricular Pacing: Effect of Myocardial Structure. \emph{Am J
          Physiol Heart Circ Physiol} April 294(4), H1753--1766, 2008.
    \end{enumerate}

  \item [Electrocardiology of acute ischemia:] Myocardial ischemia remains
    a leading cause of mortality and morbidity and has been a physiological
    focus of my research career.  We have applied inverse
    electrocardiography (ECG Imaging) to identify and localize ischemia in
    humans; we have carried out animal research studies to characterize
    the electrical changes induced by ischemia at the tissue and whole
    heart scale; we have proposed novel concepts of the location of acute
    ischemia and its evolution over time that contradict long held
    beliefs. 
    \begin{enumerate}
      \item K. Aras, B. Burton, D. Swenson, and R. MacLeod, Sensitivity of
        epicardial electrical markers to acute ischemia detection.
        \emph{J. Electrocardiol.} 47(6): 836-41, 2014.  PMCID: PMC4252649.
      \item S, Shome, R.L. Lux, B.B. Punske, and R.S. MacLeod.  Ischemic
        preconditioning protects against arrhythmogenesis through
        maintenance of both active as well as passive electrical properties
        in ischemic canine hearts. \emph{J Electrocardiol.} 40(6):S150-9,
        2007.
      \item Shibaji Shome and Rob MacLeod. Characterization of the
        transmural myocardial electrocardiographic response in in vivo
        canine working hearts under reduced flow and increased heart rate
        \emph{J Electrocardiol.}  40(4):S5-6, 2007.
      \item J.G. Stinstra, S. Shome, B. Hopenfeld, R.S. MacLeod. Modeling
        the passive cardiac conductivity during
        ischemia. \emph{Med. Biol. Eng.  Comput.} 43(6): 776-782, 2005.
    \end{enumerate}

  \item [Image analysis for cardiac arrhythmias:] MRI imaging approaches
    have made enormous progress in evaluating the substrate changes that
    arise in patients suffering from cardiac arrhythmias. We have developed
    novel approaches to analyze cardiac MRI in patients with atrial
    fibrillation, including the ability to predict treatment outcome and
    thus guide clinical management of this very common arrhythmia.  Ongoing
    studies are expanding these capabilities to evaluate the effectiveness
    of ablation strategies and determine the mechanistic relationships
    between tissue pathology and the nature of the resulting arrhythmias. 

    \begin{enumerate}
      \item B.R. Parmar, T.R. Jarrett, N.S. Burgon, E.G. Kholmovski,
        N.W. Akoum, N. Hu, R.S. MacLeod, N.F. Marrouche, and
        R. Ranjan. Comparison of Left Atrial Area Marked Ablated in
        Electroanatomical Maps with Scar in MRI \emph{J
          Cardiovasc. Electrophys.} 25(5):457--463, 2014. PMCID:
        PMC4090328.
      \item C. McGann, N. Akoum, A. Patel, E.  Kholmovski, P. Revelo,
        K. Damal, B. Wilson, J.  Cates, A. Harrison, R. Ranjan,
        N.S. Burgon, T. Greene, D. Kim, E.V.R. DiBella, D. Parker,
        R.S. MacLeod, and N.F. Marrouche.  Atrial Fibrillation Ablation
        Outcome Is Predicted by Left Atrial Remodeling on MRI. \emph{Circ A
          \& E}, Feb 1;7(1):23-30, 2014.  PMCID: PMC4086672. 
      \item R. Karim, Y. Gao, A. Tannenbaum, D.  Rueckert, J. Cates,
        T. Schaeffter, D. Peters, R.S. MacLeod, and K. Rhode.  Evaluation
        of current algorithms for segmentation of scar tissue from late
        Gadolinium enhancement cardiovascular magnetic resonance of the
        left atrium: an open-access grand challenge. \emph{J Cardiovasc
          Magn Reson} 15:105, 2013.  PMCID: PMC3878126.

      \item N. Akoum; C. McGann; G. Vergara; T. Badger; R. Ranjan;
        C. Mahnkopf; E. Kholmovski, R.S. MacLeod and N.F.
        Marrouche. Atrial Fibrosis Quantified Using Late Gadolinium
        Enhancement MRI is Associated with Sinus Node Dysfunction Requiring
        Pacemaker Implant. \emph{J. Cardiovasc. Electrophys.} 23(1):44-50,
        2012.

    \end{enumerate}

  \item[List of Published Work:] To see a full list of my publications,
    please visit the SCI Institute publications cite (and carry out a
    search for ``MacLeod'') at \url{http://www.sci.utah.edu/scipubs.html}.

\end{description}


\subsection*{D. Research Support}


\subsubsection*{Ongoing Research Support}

\noindent
\parbox{\columnwidth}{
2 P41 RR12553 (Johnson) Date: 9/1/15--8/30/2020\\
NIH/NCRR Center for Integrated Biomedical Computation\\
Role: Co-PI \\
This Center is national research resource (P-41) that has existed since
1999. The goals of the current edition are to address biomedical research
problems in bioelectric fields, imaged-based anatomy, multi-scale tissue
modeling and simulation, and scientific visualization. We will accomplish
these goals by creating state-of-the- art computational techniques and
innovative, well-engineered software, which, in combination with and freely
distributed to the science community, will significantly advance biomedical
computing research.}


\medskip

\noindent
\parbox{\columnwidth}{
  British Heart Foundation PG/15/8/31130 (Aslanidi) Date:
  5/1/2015--4/30/2018\\ 
  Dissecting multifactorial mechanisms of atrial fibrillation: Predictive
  modelling framework for evaluating medical treatments \\
  Role: Co-Investigator \\
  This project will investigate AF genesis and treatments using in silico
  3D canine atria models validated by canine experimental models of AF. The
  3D atria models will integrate major structural and functional factors
  associated with AF − fibrosis, innervation and electrical
  remodelling. They will be applied to dissect (i) multifactorial
  mechanisms of electrical activations sustaining AF, (ii) typical
  locations of electrical drivers during AF progression, and (iii) optimal
  drug and ablation treatments that can terminate various AF
  scenarios. Model predictions will be applied for evaluating key factors
  underlying successful treatments in two retrospective cohorts of AF
  patients. Hence, the genesis of predictive models in silico will assist
  in evaluating medical strategies.}


\medskip

\noindent
\parbox{\columnwidth}{
No ID Number (MacLeod)  Date: 7/1/13--6/30/17\\
Nora Eccles Treadwell Foundation \\
Role: PI\\
Title: Electrocardiographic Characterization of Myocardial Ischemia \\
The overarching goal of this research is to recognize and diagnose all
forms of myocardial ischemia using the ECG.  The overarching hypothesis of
this project is that current electrocardiographic descriptions of
myocardial ischemia are incomplete and that, as a result, the clinical
interpretation of the electrocardiogram fails to achieve its potential as a
diagnostic and monitoring approach.  We seek, therefore, through
measurements, experiments, and simulations to achieve the most complete
understanding possible of the full spectrum of ischemia in terms of its
bioelectric source, its electrical interaction with surrounding tissues,
and its reflection on the body surface.}

\medskip

\noindent
\parbox{\columnwidth}{
  R25GM107009-01 (MacLeod, Weiss, Whitaker)  Date: 7/1/2013--6/30/2016\\
  Image Based Modeling, Simulation, and Visualization
  Summer Course for Biomedical Researchers\\
  Role: Co-PI\\
  The goal of this proposed program is to expand the scope of current
  Center for Integrative Biomedical Computing (CIBC) and Muskuloskeletal
  Research Lab (MRL) training to create a dedicated two-week course in the
  area of image based modeling and simulation applied to bioelectricity and
  biomechanics.}

\medskip

\noindent
\parbox{\columnwidth}{
  R25GM107009-01 (Aylward, Cates)  Date: 7/1/2013--6/30/2016\\
  WEB-Based Infrastructure for Comparison and
  Validation of Image Computing Methods\\
  Role: Co-Investigator\\
  The goal of this project is to develop the infrastructure for and deploy
  a commercial installation of an Algorithm Evaluation Service (AES) to
  help bridge the gap between algorithm researchers and commercial product
  developers. We will  support a novel
  mechanism for algorithm submission based on virtual machine technology
  that addresses clinical integration, security, and computational resource
  scalability to support extensive testing. We will  make
  the submission of algorithms to  challenges an inherent and effortless
  part of algorithm development for researchers. We will validate
  the resulting system using additional grand challenges, and we will
  deliver it to and receive feedback from our first commercial customer as
  part of the proposed work.}

\medskip

\noindent
\parbox{\columnwidth}{
  Siemens Medical Solutions (MacLeod)  Date: 09/12/2012--12/31/2015\\
  Overlay and registration of MRI-derived scar maps to real-time
  fluoroscopy for repeat atrial fibrillation (AF) ablation\\
  Role: PI\\
  This is an industry funded project to develop a means to merge previously
  acquire MRI images and content derived from those images with real time
  fluoroscopy displays in the catheterization laboratory for atrial
  ablation treatments. The goal of the project will be to develop a
  combination of custom and available software tools that will perform all
  the necessary steps of registering (aligning) the scar map with
  fluoroscopic images captured from fluoroscopy performed as part of a
  repeat ablation procedure.  }


\noindent
\subsubsection*{Recently Completed Research Support}

\noindent
\parbox{\columnwidth}{
2 P41 RR12553 (Johnson) Date: 9/1/2010--8/30/2015\\
NIH/NCRR Center for Integrated Biomedical Computation\\
Role: Co-Investigator and Center Co-director \\
This Center is national research resource (P-41) that has existed since
1999. The goals of the current edition are to address biomedical research
problems in bioelectric fields, imaged-based anatomy, multi-scale tissue
modeling and simulation, and scientific visualization. We will accomplish
these goals by creating state-of-the- art computational techniques and
innovative, well-engineered software, which, in combination with and freely
distributed to the science community, will significantly advance biomedical
computing research.}

\medskip

\noindent
\parbox{\columnwidth}{
U54E B005149-06 (Kikinis)  Date: 10/1/10--9/30/14\\
NIH/Roadmap Initiative National Alliance for Medical Image Computing: DBP
Atrial Fibrillation\\
Role: Co-Investigator and DBP PI\\
NA-MIC is a multi-institutional, interdisciplinary team of computer
scientists, software engineers, and medical investigators who develop
computational tools for the analysis and visualization of medical image
data. The purpose of the Center is to provide the infrastructure and
environment for the development of computational algorithms and open-source
technologies, and then oversee the training and dissemination of these
tools to the medical research community.}


\end{document}
