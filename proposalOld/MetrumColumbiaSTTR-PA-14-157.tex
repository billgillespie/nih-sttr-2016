\documentclass[11pt]{nih2016}
%\documentclass[10pt]{article}
%\usepackage{onr}

\usepackage{natbib}
\usepackage[colorlinks,breaklinks,pdftitle={ONR Proposal},pdfauthor={Gelman, Andrew et al.}]{hyperref}
\hypersetup{linkcolor=black,citecolor=black,filecolor=black,urlcolor=black}

\newcommand{\myurl}[1]{\href{#1}{{\tt #1}}}%{{\tt #1}} % 

\begin{document}

\begin{center}
\Large\bf Fast and flexible Bayesian platform for pharmacometric
analysis \\
\ \\
Volume Two: Technical Proposal
\end{center}

\section{Identification and Significance of the Problem or Opportunity.}

% Define the specific technical problem or opportunity addressed and
% its importance. (one page)

Nonlinear ordinary differential equations (ODEs) are widely used for modeling
the dynamics of complex systems in the physical, biological, and
social sciences, as well as across the engineering disciplines. Given
(noisy) measurements of system inputs and outputs, researchers are
faced with the ``inverse problem'' of estimating parameters from data
relative to a statistical model of the phenomenon at hand and the
measurement process.  Examples of application areas include
drug-disease response in pharmacology/toxicology
\cite{peterson-rigg:2010, gelman-et-al:1996}, thermal conductivity and
heat transfer for rocketry, airflow analysis over control surfaces in
in aeronautical engineering \cite{alifanov:2012}, facial recognition
in computer vision \cite{aubert-kornprobst:2006, bovik:2010}, dynamics
of gaseous bodies with respect to background magnetic fields in
astrophysics \cite{tobias-et-al:2011}, population diffusion and
spatial dynamics in ecology \cite{gopalsamy:2013}, continuous-time
asset pricing in econometrics \cite{johannes-polson:2010}, soil-carbon
respiration in biogeochemistry \cite{manzoni-porporato:2009}, cellular
regulation in systems biology \cite{baron-gastonguay:2015,
  baron-et-al:2013, leclerc-et-al:2016}, fatigue in land and sea
combat \cite{rubio-campillo:2016}, and pressure changes and oceanic
flow in fluid dynamics for meteorology \cite{charney-phillips:1953},
to name but a few.

Inference for ODEs can be challenging, both from a statistical
perspective (the solution space of a nonlinear inverse problem can
have a complex geometric structure, far from the Gaussian
distributions used in standard asymptotic inference) and also
computationally: solving the inverse problem in a probabilistic sense
requires tracing out the space of parameter values that are consistent
with data and prior information.  Even the simplest ODEs generally do
not admit closed-form solutions, hence computational approaches are
necessarily iterative, requiring a search through parameter space.
Such a search is most effectively performed using gradients, which in
turn raises the challenging problem of computing gradients of
differential equation solutions. In this setting, problems of
parameter estimation, prediction, uncertainty quantification, and
hence decision-making under uncertainty, are intractable analytically
and notoriously difficult to solve numerically.

Stan \cite{carpenter-et-al:2016, stan-development-team:2016,
  mcelreath:2016} is a widely used, open-source, probabilistic
programming language and Bayesian inference engine.  Stan currently
provides state-of-the-art Bayesian inference algorithms for inverse
problems based on gradients.  Full Bayesian inference is performed
exactly (up to some specified precision) using Hamiltonian Monte Carlo
(HMC, \cite{neal:2011}), a Markov chain Monte Carlo (MCMC) method for
sampling from the posterior and computing expectations; approximate
algorithms, which can provide exploratory analysis and even useful
estimates at orders of magitude greater throughput, include mean-field
and full-rank variational inference (VI,
\cite{wainwright-jordan:2008}) and simple optimization-based Laplace
approximation \cite{gelman-et-al:2013}.  Stan includes a simple
Runge-Kutta initial-value solver for ordinary differential equations.
% , but the solver is not optimized and does not support stiff
% differential equations, and the interface provided by Stan is at a
% very low level of core vector operations.

We propose to extend Stan to incorporate state-of-the-art solvers for
ordinary differential equations and differential algebraic equations
into its inference procedures.  Based on the performance of our
non-stiff solvers, we believe we can achieve an order of magnitude or
more speedup over the existing state-of-the-art, due to Stan's
built-in efficient automatic differentiation library and efficient
samplers for full Bayesian inference and optimizers for approximation
Bayes through variational inference and maximum marginal likelihood.
Stan provides the means to create coupled systems of differential
equations through nested automatic differentiation, which provide full
sensitivities (derivatives of solutions with respect to parameters)
\cite{lee-hovland:2002,serban-hindmarsh:2003,carpenter-et-al:2015} and
thus allow Stan's derivative-based inference engines to be used with
statistical models involving non-linear differential equations.  The
key for our proposed work will be calculating the sparse, structured
Jacobians of these coupled systems to permit solutions of stiff
systems of equations (typically characterized by varying time scales
among the components, such as the varying absorption, distribution,
metabolism, and excretion of toxins in bone, fat, blood, and organs
such as kidneys).

We will also extend Stan to deal with events arising from external
inputs such as discrete (bolus) dosing in pharmacology or leaf-litter
decomposition by enzymes in soil-carbon modeling. These represent
general improvements in the Stan language which should serve specific
customers in the present commercial development.  In that way, we
propose at Columbia University to serve the general science and
engineering community with improvements in open-source software while
at the same time at Metrum producing a product and services for which
there is a strong market.

For concreteness, we will evaluate the tools produced using
pharmacometric data with a range of sophisticated statistical and
mathematical models in common use \citep[e.g.,][]{ette-williams:2007,
  schmidt-derendorf:2014}. Our goal is to extend the size of data
sets, sophistication of models (e.g., varying effects by patient,
missing data, meta-analysis of multiple drug studies and placebo
controls, increased granularity of spatio-temporal modeling,
hierarchical modeling of (sub)populations), and speed of solvers by an
order of magnitude or more over the state of the art, a target we
believe is realistic given Stan's performance with simple non-stiff
ODE solvers \cite{weber-et-al:2014}.  The result will be an even more
flexible Bayesian statistics platform that supports analysis of
heterogeneous collections of data conditioned on models of great
stochastic and deterministic complexity and quantitative prior
knowledge, leading to better calibrated and sharper (more precise)
predictions \citep[see, e.g.,][]{gneiting-et-al:2007}.

\subsection{Background}

Modern research in science and engineering is characterized by big,
messy data. Scientific datasets are not just bigger in that there are
more instances of the same thing---a mere increase in sample size
would make our job as data analysts easier. Rather, there is more {\em
  breadth} and {\em complexity} to the data: more subgroups,
locations, or time granularity than is currently being modeled, more
partial and noisy measurements that cannot easily be incorporated into
standard models, more related studies available for meta-analysis and
prior formulation, more information on the physical mechanisms of
measurement, more information on the population units being measured,
and more fine-grained information on the predictions desired.

Consequently, in order to utilize big data we need complementary big
models that can, for example, adjust the data to match sample to
population, to match treatment to control groups in causal inference,
or learn about interactions and variation and make individualized
predictions rather than be limited to averages.  Once we have big
models we inevitably have multiple levels of uncertainty and
variation, which can be studied using Bayesian inference, as is
demonstrated in a large literature of applied statistics, including
our own books and research articles. The big computational challenge
is scalability: developing algorithms and implementations that work
well and do not require too much storage and computation time as
datasets become larger and the amount of predictive information
increases.

\subsection{Modeling and simulation in drug research and development}

Pharmacometric data typically consists of longitudinal measurements
gathered from different sources such as drug concentrations in blood
plasma and pharmacological effects often resulting from a sequence of
treatment events, e.g., drug doses. Mathematical modeling and
simulation facilitates decision-making and risk-benefit assessment by
drug developers regulators, health care providers, and patients.

Mathematical models describing measurements over time are often
compartmental models formulated as systems of first order ordinary
differential equations (ODEs). These models often also involve a
hierarchy of random effects required to describe sources of
variability, e.g., inter-individual, inter-occasion and residual
variation. Highly complex systems pharmacology models describe
physiologic, biochemical, and pharmacological processes at multiple
scales (molecular, cellular, tissue, organ, organism and population)
\citep[e.g.,][]{baron-et-al:2013, peterson-rigg:2010}. Even relatively
simple compartmental models used for pharmacokinetic data analysis
require solution of systems of ODEs often by computationally demanding
numerical methods.

Although there are well-developed methods for solving continuous ODEs
given initial values, tools for inference in models that incorporate
ODEs as part of the statistical model are less-developed. Discrete
events such as bolus doses into a model compartment induce
discontinuities in the ODEs or their solution, which adds additional
computational complexity. We propose to evaluate and extend software
tools to incorporate better methods for solving ODEs and handle
discrete events.

Systems pharmacology models are very useful for applications in
translational drug research--predictions of drug disposition and
effects in humans based on preclinical research. The development of
such models is highly challenging and resource intensive. It requires
integration of large amounts of prior biochemical and physiological
knowledge with a heterogeneous collection of data from in vitro,
animal and human studies. Bayesian methods provide a rational approach
to the model-based analysis of data conditioned on such complex models
and quantitative problem-specific knowledge about the model parameters
\citep[e.g.,][]{gelman-et-al:1996, leclerc-et-al:2016}.

%The addition of numerical analytical methods for linear and nonlinear
%systems of differential and algebraic equation to Stan will expand its
%range to include models encountered in many scientific and engineering
%fields.

\section{Phase I Technical Objectives.}

% Enumerate the specific objectives of the Phase I work, including the
% questions the research and development effort will try to answer to
% determine the feasibility of the proposed approach.

The overall objective of this proposal is to develop additional
functionality within Stan to support pharmacometric applications. This
includes the addition of functions for implementing compartmental PK/PD
models and schedules of dosing and other discrete events. The
resulting enhancements to Stan will also support other applications
that would benefit from Bayesian data analysis using complex models
requiring numerical solution of systems of ordinary differential or
algebraic equations.

\subsection{Technical Objectives}

The combined scope of the proposed Phase I and Phase II work will
result in a pharmacometric modeling and simulation platform that is
more flexible and that provides order(s) of magnitude more scalable
and more efficient Bayesian computation than available tools. This
will be accomplished by extending the capabilities of Stan to include
support for pharmacometric models. The goal is a software tool that
combines 

\begin{itemize}
\item numerical solution of systems of ordinary differential equations
  (ODEs),
\item numerical solution of systems of algebraic equations,
\item Bayesian methodology that permits integration of new data with
  prior information,
\item efficient computation of posterior densities and expectations
  with respect to them (e.g., for event probability estimation and
  prediction),
\item a flexible model specification language that permits
\begin{itemize}
\item probabilistic models that allow multiple levels and sources of
  variability and uncertainty,
\item implementation of complex models that combine submodels with
  different stochastic hierarchies. E.g., translational models that
  combine linked submodels for simultaneous analysis of clinical,
  animal and in vitro experiments,
\item model calculations that depend on a
  potentially complicated sequence of events that affect the model
  predictions (e.g., multiple dosing, changes in non-drug treatments,
  time-varying model parameters).
\end{itemize}
\end{itemize}

\subsection{Comparison to Existing Tools}\label{existing}

Various existing software tools address subsets of those features, but
none address them all. Shortcomings of currently available software
are outlined below. Commercial tools that specialize in pharmacometric
applications such as NONMEM, Monolix and Phoenix offer

\begin{itemize}
\item Good support for compartmental models based on ODEs and for
  typical schedules of dosing and other discrete events.
\item Including ODE solvers specialized for stiff, non-stiff and
  linear ODEs
\item Limited flexibility w.r.t. probabilistic model components.
\item NONMEM \& Monolix employ relatively inefficient MCMC algorithms
  for Bayesian analyses.
\item Phoenix provides no support for Bayesian analysis.
\item Stochastic structure of supported models is limited:
\item Only Normal-Wishart priors
\item Only normally distributed variability (except for residual
  variability)
\item Cannot combine submodels with different stochastic hierarchies
\end{itemize}

Stan is in part modeled on the previous state-of-the-art tool for
Bayesian inference via MCMC, WinBUGS \cite{lunn-et-al:2000}, which
contains a module PKBugs for ODE solving, and for which we have the
BUGSModelLibrary of sample models for testing
\cite{gastonguay-et-al:2010}.  WinBUGS and other Gibbs-sampling based
approaches provide

\begin{itemize}
\item inefficient MCMC algorithms compared to Stan, which is
\begin{itemize}
\item  two orders of magnitude more scalable, and
\item  between two times faster and orders of magnitude faster,
  with greater speedups in more challenging problems.
\end{itemize}
\end{itemize}

\subsection{Advantages of Stan}

In contrast, Stan provides

\begin{itemize}
\item more efficient posterior simulation algorithms (HMC/NUTS) than
  other pharmacometrics and general purpose Bayesian analysis tools,
\item approximate Bayesian (ADVI) and optimization methods,
\item numerical solvers for non-stiff ODEs,
\item only limited, primitives-based support for pharmacometrics models,
\item no tools for handling typical schedules of dosing and other discrete events,
\item a single simple ODE solver (no stiff solvers, no efficient
  solvers for linear ODEs), and
\item no tools for solving nonlinear algebraic equations, which are
  required for steady-state solutions for ODES, a commonly encountered
  challenge in pharmacometric analyses.
\end{itemize}

These properties of Stan make it a highly attractive general purpose
program for Bayesian data analysis and arguably the best available
platform upon which to build additional features. As outlined above,
Stan lacks some important capabilities required for many
pharmacometric applications.


\subsection{Objectives for Phase I Base}

\begin{itemize}
\item Detailed planning and feasibility assessment for development of
  a suite of Stan enhancements to support pharmacometric modeling and
  simulation
\item Development of Stan components for non-steady-state calculations
  for compartmental PK/PD models
\item Handling of discrete events, e.g., bolus or constant rate input,
  piecewise constant parameters,
\item Implementation of functions for analytic solutions for standard
  linear one and two compartment PK models with or without first order
  absorption.
\item Functions that integrate numerical solution of ODEs with the
  event handler
\item Non-stiff ODEs solved with Adams-Moulton or Runge-Kutta methods
\item Stiff ODEs solved with Backward Differentiation Formulas (BDF)
\end{itemize}

\subsection{Objectives for Phase I Option}

\begin{itemize}
\item R package for PK/PD data handling and model implementation.
\item Linear ODEs solved semi-analytically using matrix exponentials
\item Function that integrates numerical solution of linear ODEs
  with the event handler
\end{itemize}

\section{Phase I Statement of Work.}

% (Objectives and Statement of Work (sections 3 and 4), 10-12 pages)

% (a) Provide an explicit, detailed description of the Phase I
% approach. For the Phase I option, describe appropriate research
% activities which would commence at the end of Phase I should the
% Navy elect to exercise the option. The Statement of Work should
% indicate what tasks are planned, how and where the work will be
% conducted, a schedule of major events, and the final product(s) to
% be delivered. The Phase I effort should attempt to determine the
% technical feasibility of the proposed concept. The methods planned
% to achieve each objective or task should be discussed explicitly and
% in detail. This section should be a substantial portion of the
% Technical Volume section.

% (b) Due to the short timeframe associated with Phase I of the STTR
% process, the Navy does not recommend the submission of Phase I
% proposals that require the use of Human Subjects, Animal Testing, or
% Recombinant DNA. This solicitation may contain topics that have been
% identified by the Program Manager as research or activities
% involving Human /Animal Subjects and/or Recombinant DNA. In the
% event that Phase I performance includes performance of these kinds
% of research or activities, please identify the applicable protocols
% and how those protocols will be followed during Phase I. Please note
% that funds cannot be released or used on any portion of the project
% involving human/animal subjects or recombinant DNA research or
% activities until all of the proper approvals have been obtained.
% (see DoD 2013.1 SBIR Solicitation Sections 4.7 – 4.9).

\subsection{Phase I Base}

Phase I Base will be limited to (1) detailed planning and feasibility
assessment for development of a suite of Stan enhancements to support
pharmacometric modeling and simulation, and (2) developing an initial
subset of the planned components.

\subsubsection{Planning and feasibility assessment}

The first stage of the work will be to develop more detailed
functional specifications for Stan components that fulfill the
following general functional requirements: Functions that calculate
the amount in each compartment of a compartmental PK/PD model at a
specified time given the amount in those compartments at a previous
time, Calculations based on analytic solutions for the standard linear
one and two compartment PK models with or without first order
absorption. Calculations based on numerical solution of a
user-specified system of ordinary differential equations (ODEs).
Specific functions will be required for different categories of ODEs
\citep{byrne-hindmarsh:1975}:

\begin{itemize}
\item Stiff ODEs solved numerically via methods based on backward
  differentiation formulas (BDF).
\item Non-stiff differential equations solved numerically via
  Adams-Moulton or Runge-Kutta methods.
\item Linear ODEs solved semi-analytically via matrix exponential methods.
\item Functions that calculate the amount in each compartment of a
  compartmental PK/PD model at a vector of specified times given a
  schedule of discrete events, e.g., bolus or constant rate doses and
  piecewise constant parameters. These functions will be developed by
  integrating the functions for calculating the compartment amounts at
  a single time with software for handling the event schedule. 
\item One or more functions for numerical calculation of the solution
  to a system of nonlinear algebraic equations.
\item Functions that calculate the amount in each compartment of a
  compartmental PK/PD model at a vector of specified times under
  quasi-steady-state conditions resulting from a periodic input, e.g.,
  repeated administration of equal doses at equally spaced time
  intervals. This requires numerical solution of both ODEs and
  algebraic equations in order to solve a boundary value problem.
\item Integral with the development of the above functional
  specifications will be assessing the feasibility of implementing
  them within Stan. Feasibility assessment may also extend to
  consideration of additional functionality that may be desirable for
  pharmacometric and other scientific modeling applications, e.g.,
  solutions of delay differential equations, stochastic differential
  equations, differential algebraic equations and partial differential
  equations.
\end{itemize}

\subsubsection{Development of Stan components for non-steady-state
  calculations for compartmental PK/PD models} 

The software developed in this project will be written in C++. The C++
code will result in new functions in the Stan model specification
language. Subject to revision based on the planning and feasibility
effort described above, C++ code consistent with Stan architecture
will be developed for the following tasks. A function will be
constructed that calculates the solution of a system of ODEs at a
specified sequence of times given a schedule of discrete events, e.g.,
bolus or constant rate input or piecewise constant parameters. This
will be a general event handler that calls another function that
returns the solution to a specific ODE system at a single time given
initial values, i.e., implementing a recursive approach in which the
ODEs are solved over each continuous interval given initial conditions
for that interval calculated from the solution for the prior
interval. Functions for calculating analytic solutions for standard
linear one and two compartment PK models with or without first order
absorption will be created.  A function for calculating numerical
solutions of non-stiff ODEs using an Adams-Moulton method will also be
implemented. This will probably employ code from the Sundials
collection \citep{hindmarsh-et-al:2005}. Stan functions have already
been developed for calculating numerical solutions of ODEs using
Backward Differentiation Formulas (BDF) for stiff ODEs and Runge-Kutta
methods for non-stiff ODEs.

Additional C++ code will be written that integrates the ODE solution
functions with the event handler and exposes the results as functions
in the Stan language. In addition a general purpose ODE solver
function using the Adams-Moulton method will be added to Stan. The function
will be consistent with the current Stan implementations of the
Runge-Kutta and BDF solvers.

\subsubsection{Software development practices}

Stan is developed using current best-practices for production software
development.  These rely on a combination of test-driven interface
design, established coding standards, thorough code reviews,
exhaustive unit tests, and ongoing continous integration testing
on all supported platforms.

Stan's software distributions, including the source version control
repositories and the official releases, along with the Wiki used for
design documents, and the issue tracker for feature requests and bug
reports, are hosted by GitHub.%
%
\footnote{\myurl{http://github.com/}}
%
The entire distribution and the in-development code are open access,
and available through the public-facing GitHub web site.  All work is
carried out in branches and submitted to be merged with the
development branch through pull requests which are openly code
reviewed.  All behavior must be thoroughly tested, both with unit
tests for the local functionality and integration tests to guard
against breaking things end-to-end with local changes.


\subsubsection{User-facing changes}

Stan has over 1500 registered users on its users list and has seen
over 200 academic publications across science, engineering, business,
and sports.  The results of this project will involve extensive
additions to the user-facing documentation. This includes
interface-by-interface walk-through tutorials and example models. Stan
is increasingly being included in third-party textbooks, and we strive
for backward compatibility so that these references are not made
obsolete. Stan also has a dedicated web site, which is also managed
through GitHub.

\subsubsection{Collaboration plan}

The project will be a close collaboration between personnel at Metrum
Research Group and the Stan development team based at Columbia
University. Metrum scientists provide expertise in pharmacometrics. In
most cases they will be responsible for initial functional
specifications and will often develop prototype code for the desired
functionality in Stan. The Stan team will collaborate with Metrum
scientists to refine the functional specifications. They will adapt or
replace the prototype code to develop Stan code suitable for inclusion
in publicly released versions of Stan. Part of that process will
include rigorous testing to assure the code functions as intended.

The main coordination mechanisms we use are GitHub for version control
and code review, Jenkins for continuous integration testing, weekly
Google+ hangouts for online meetings as well as additional meetings as
needed.  In this particular case, we are within public transportation
distance of each other, which will facilitate regular face-to-face
meetings with whiteboards.


\subsection{Phase I Option}

\subsubsection{R package development}

R (\url{https://www.r-project.org/}) is a statistics software package widely used by
pharmacometricians and other scientific fields. Users can develop
additional R components and make them available as easily installed
packages. An R package will be developed to provide a more
familiar and user-friendly tool for the following tasks:

\begin{itemize}
\item Translation of data sets formatted for NONMEM
  (\url{http://www.iconplc.com/innovation/solutions/nonmem/}) into
  Stan-compatible formats
\item Model specification
\item Model execution using either the rstan or cmdstan interfaces to
  Stan
\item Tabular and graphical analyses of Stan output specialized for
  pharmacometric applications
\item The component functions will be programmed in the R language.
  They will be made available to the community via open distribution
  sites, e.g., CRAN (\url{https://cran.r-project.org/}) or R-Forge
  (\url{https://r-forge.r-project.org/}).
\end{itemize}

\subsubsection{Development of additional Stan components for linear compartmental PK/PD models}

A large fraction of pharmacometric models are based on systems of
linear ODEs with constant coefficients. Such models can be solved
semi-analytically via a matrix exponential approach. A matrix
exponential method can solve applicable ODEs much more rapidly than
the multi-step numerical methods implemented as part of Phase I Base.
A Stan function for calculating non-steady-state and steady-state
solutions of linear ODEs using matrix exponentials will be developed.
The specific computational approach will be identified during the
Phase I planning. A variety of algorithms have been proposed
\citep{moler-vanloan:2003} and implemented in various languages, e.g.,
FORTRAN \citep{sidje:1998} and C++
(\url{https://people.sc.fsu.edu/~jburkardt/cpp_src/matrix_exponential/matrix_exponential.html}).

As with the ODE solvers described in section 3.1.2, a Stan function
integrating the matrix exponential solver with the event handler will
be developed.

\section{Related Work.}

% (one page)
% Describe significant activities directly related to the proposed
% effort, including any conducted by the principal investigator, the
% proposing firm, consultants, or others. Describe how these
% activities interface with the proposed project and discuss any
% planned coordination with outside sources. The technical volume must
% persuade reviewers of the proposer's awareness of the state-
% of-the-art in the specific topic. Describe previous work not
% directly related to the proposed effort but similar. Provide the
% following: (1) a short description, (2) the client for which work
% was performed (including the individual to be contacted and phone
% number), and (3) date of completion.

\subsection{Metrum Research Group}

Metrum Research Group, established in 2004, is an innovative provider
of pharmacometric modeling and simulation services and solutions to
support biomedical decision-making. Metrum scientists employ complex
pharmaco-statistical models to analyze data arising from pre-clinical,
clinical, and post-marketing studies. The models used typically
include a hierarchy of random effects to describe various sources of
variability, e.g., inter-individual and residual variability. Often
they describe longitudinal outcomes using the solutions of systems of
nonlinear differential equations. In other words, Metrum scientists
regularly work with the kinds of models this proposal seeks to
implement in Stan.

In addition to the modeling and simulation services, Metrum has
developed and distributed open source software tools to facilitate
pharmacometric modeling tasks. This includes the R packages metrumrg
\citep{bergsma-et-al:2013} and mrgsolve
\citep{baron-gastonguay:2015}. Dr. Gillespie, the principal
investigator, developed BUGSModelLibrary
\citep{gillespie-gastonguay:2009}, programs that augment the Bayesian
modeling platform WinBUGS \citep{lunn-et-al:2000} with functions for
pharmacometric modeling. This includes functions for handling
schedules of discrete events, e.g., dosing, and numerical solution of
differential and algebraic equations.

Metrum also has experience with commercial software development and
marketing. Metrum offers a high performance cloud computing
platform-as-a-service called Metworx
\url{http://metrumrg.com/metworx.html} that provides users with
managed, on-demand grid computing for as many environments and cores
as needed.

\subsection{Andrew Gelman and the Stan Development Team}

\subsubsection{Core Stan Platform}

Carpenter, Gelman, and Lee have been core contributors to the design,
implementation, documentation, and maintenance of Stan since its
inception in January 2011.  Gelman provides the statistical direction,
and Carpenter and Lee are the two full-time software developers.  This
work was initially funded in part through two grants:
%
\begin{itemize}
\item Department of Energy  (DE-SC0002099 Petascale Computing)
\item National Science Foundation (CNS-1205516: Stan: Scalable
  Software for Bayesian Modeling) 
\end{itemize}
%
It is currently funded in part through the following grants:
%
\begin{itemize}
\item Alfred P. Sloan Foundation 
(G-2015-13987: Stan Community and Continuity)
\item Office of Naval Research 
(Informative Priors for Bayesian Inference and Regularization)
\item 
Institute of Education Sciences 
(Statistical and Research Methodology: Solving Difficult Bayesian
Computation Problems in Education Research Using Stan)
\end{itemize}


\subsubsection{ODEs in Stan}

Carpenter, Gelman, and Lee have worked for Novartis AG as consultants
to develop the capability to solve initial value problems for ordinary
differential equations in Stan and to work on methodology for
drug-disease PK/PD models including multilevel models for patient
populations and meta-analysis.  This work was funded by two groups
within Novartis; contacts are
%
\begin{itemize}
\item Novartis, Switzerland: Dr.\ Sebastian Weber (+41 61 324 6217)
\item Novartis, United States: Dr.\ Wenping Wang (+1 862 778 9009).
\end{itemize}
%
The first short contract was completed in December 2013 and there are
two contracts currently ongoing.

\section{Relationship with Future Research or Research and
  Development.}

%  (half page)

% (a) State the anticipated results of the proposed approach if the
% project is successful.

% (b) Discuss the significance of the Phase I effort in providing a
% foundation for a Phase II research or research and development
% effort.

% (c) Identify the applicable clearances, certifications and approvals
% required to conduct Phase II testing and outline the plan for
% ensuring timely completion of said authorizations in support of
% Phase II research or research and development effort.
% (Bill)

The Stan functions developed in Phase I are both valuable deliverables
in their own right and foundations upon which additional functionality
will be developed in Phase II. For example the ODE solver and event
handler functions developed in Phase I will be used again in the
development of functions for performing steady-state calculations.
Probable Phase II objectives include development of Stan functions
for:

\begin{itemize}
\item Numerical solution of nonlinear algebraic equations
\item Calculation of the amount in each compartment of a compartmental
  PK/PD model at a vector of specified times under quasi-steady-state
  conditions resulting from a periodic input, e.g., repeated
  administration of equal doses at equally spaced time intervals. This
  requires numerical solution of both ODEs and algebraic equations in
  order to solve a boundary value problem.
\item Automatic selection between the BDF and Adams-Moulton solvers
  based on assessment of ODE system stiffness
\item Within chain parallel computation for hierarchical model
  structures commonly used in pharmacometrics
\end{itemize}

Other potential Phase II objectives include development of functions
for numerical solution of delay differential equations, stochastic
differential equations, differential algebraic equations and partial
differential equations. All of the aforementioned Phase II objectives
would build on the code developed in Phase I.


\section{Commercialization Strategy.}

% 1 to 2 pages
% Describe in approximately one page your company’s strategy for
% commercializing this technology in DoD, other Federal Agencies,
% and/or private sector markets. Provide specific information on the
% market need the technology will address and the size of the market.
% Also include a schedule showing the quantitative commercialization
% results from this STTR project that your company expects to achieve.
% (one to two pages)

\noindent
{\sc Overview}
\\[2pt]
The successful development of the Stan ODE technology, will result in
the delivery of novel services and solutions to customers engaged in
data science and quantitative analytics. Given the strong track record
of Metrum Research Group in the pharmaceutical and biotechnology
industries, the commercialization strategy is primarily focused on
private sector opportunities in these markets. Extension of the
commercialization plan to defense, and other government or private
markets is also plausible, and will be discussed briefly.
\\

\noindent
{\sc Opportunity}
\\[2pt]
The challenges of identifying new therapeutic strategies for unmet
medical needs, along with intense competition, rising research and
development costs, increased regulatory scrutiny, and post-marketing
third party payor expectations, require innovators in this space to
efficiently apply available information to decision-making in the
product development cycle. Data science, modeling and simulation,
allow for a quantitive understanding of the problem. Bayesian data
analysis methods, in particular, are well suited to this sort of
quantitative decision support, in that they allow for the formal
inclusion of prior information and projections of posterior
probabilities of potential decision outcomes. As the complexity of
these models grows with increasing information content and better
mechanistic insights, the capabilities of existing analysis tools are
stretched and even exceeded.\\ 

\noindent
Tools which allow for efficient hierarchical Bayesian data analysis of
models specified as a system of differential equations are  needed to
advance the quantitative decision-making process. Few solutions exist
(see section \ref{existing} Comparison to Existing Tools), and all are
significantly limited in scope, capability, or computational
efficiency when compared to the proposed Stan ODE technology.\\ 
 
\noindent
{\sc Intellectual Property}
\\[2pt]
The Stan ODE technology will be distributed as part of the existing
open-source Stan software project \url{http://mc-stan.org/} under a
public license. This has been a common distribution strategy for other
Bayesian modeling tools such as OpenBUGS
(\url{http://www.openbugs.net}), and JAGS
(\url{http://mcmc-jags.sourceforge.net}) , the R statistical software
(\url{https://www.r-project.org}), and operating systems, such as
Linux (\url{https://www.debian.org}). Our strategy is to develop or
extend value-added intellectual property (IP) around the open-source
product, as successfully demonstrated by other commercial entities,
such as Red Hat (\url{http://www.redhat.com}), Revolution Analytics
(\url{http://www.revolutionanalytics.com}), and R Studio
(\url{https://www.rstudio.com}).\\ 

\noindent
{\sc Commercialization Context}
\\[2pt]
Metrum Research Group has provided services and solutions (nearly 300
different contracts) to more than 120 different companies and research
centers in the pharmaceutical and biotech space. This customer base
includes 13 of the 15 largest global pharmaceutical companies and 7 of
the top 10 biotech companies. The services include: 1). contract
modeling and simulation to support biomedical decision making and
regulatory filings, 2). training on the application of advanced
modeling and simulation methods and strategies, and 3). consulting on
the qualification and management of software and computational
infrastructures in regulated industries, such as
pharmaceuticals. Solutions include: 1). open-source tools for
pharmacometrics applications
\url{http://metrumrg.com/opensourcetools.html}, 2). the METAMODL
library of disease models and curated data sets
\url{http://metrumrg.com/metamodl.html}, and 3). a high performance
computing Platform-as-a-Service (PaaS) solution called METWORX
\url{http://metrumrg.com/metworx.html}. In addition to the scientific
team, the company also employs marketing, business development,
accounting, quality/compliance, information systems, project
management, and related support staff.\\ 

\noindent
The Stan ODE technology will be commercialized as both services and
solutions. This new technology will enable extension of modeling and
simulation services related to multi-scale systems pharmacology,
target-mediated drug disposition, physiologically based
pharmacokinetics, and Bayesian adaptive dosing regimens and trial
designs. New training services are also anticipated around Bayesian
modeling and in these same topic areas. Commercialization via
value-added products and solutions is also planned. These include:
software qualification packages, graphical user-interfaces, and
extended technical support. The new technology will also be integrated
into existing Metrum Research Group commercial products and
solutions. The METAMODL library will be extended to include Bayesian
hierarchical implementations of existing and new disease models. The
METWORX PaaS product, which allows for on-demand, auto-scaling,
distributed cloud computing will integrate the Stan ODE technology in
a regulatory compliant workflow, facilitating enterprise adoption of
the tool. In addition METWORX will provide the computational backbone
for between-chain and later, within-chain parallel processing of
Bayesian ODE models in Stan.\\ 

\noindent
{\sc Market Assessment}
\\[2pt]
In the translational through post-marketing phases of the
pharma/biotech product cycle, the current market for
pharmacometrics-related modeling and simulation services is estimated
at \$100 million annually. This includes the related market of
quantitative systems pharmacology (QSP), as well as the more
traditional applications of pharmacokinetic-pharmacodynamic (PK-PD)
modeling, disease progression modeling, trial simulation, and training
in all of these areas. Metrum Research Group's current market share is
approximately 7\%. In addition to the services market, a market for
pharmacometrics products and solutions has developed over the past two
decades. These products and solutions include traditional desktop or
server data analysis software, therapeutic area data/model libraries,
and a nascent market for managed and qualified high performance
computing solutions.  The current market for pharmacometrics products
and solutions is approximately \$60 million annually. Metrum Research
Group's current market share is approximately 1\%. Similar markets
exist in other quantitative fields, and the most logical targets for
expansion would be statistics/biometrics, toxicology, and systems
biology groups within the pharmaceutical/biotech sector. In addition,
the Stan ODE technology, and value-added IP will be applicable to
markets in defense, economics, and ecology. The size of these markets
has not been quantified. \\ 

\noindent
{\sc Revenue Forecast}
\\[2pt]
We estimate Metrum Research Group services, products, and solutions
revenues directly related to the Stan ODE technology of \$2 million in
the first two years of commercialization, with growth to \$30 million
annually within 10 years in the pharmaceutical/biotechnology sector
alone. We also anticipate that Stan ODE technology will facilitate
introduction and growth of the METWORX solution to new markets outside
of pharma/biotech. Revenue growth for those markets is unknown. 


\section{Key Personnel.\label{personnel}}

% Identify key personnel who will be involved in the Phase I effort
% including information on directly related education and experience.
% A concise technical resume of the principal investigator, including
% a list of relevant publications (if any), must be included (Please
% do not include Privacy Act Information). All resumes will count
% toward the applicable page limitation.

% (format for brief resume)
% INVESTIGATOR NAME
% School, Degree, Year

% RELEVANT EXPERIENCE

% Please provide a concise description of the investigator’s relevant
% technical experience and its application to this topic.

% RELEVANT AWARDS
% Please list any awards received for work related to this topic.

% RELEVANT PUBLICATIONS
% Please list any publications relevant to this topic.

\subsection{Metrum Research Group}

\noindent
{\sc William R.\ Gillespie}
\\[2pt]
The University of Iowa, Ph.D., Pharmacy, 1987 \\
The University of Michigan, M.S., Pharmacy, 1980 \\
Wayne State University, B.S., Pharmacy, 1976
\\

\noindent
{\sc Relevant Experience}
\\[2pt]
For nearly 30 years Bill has been involved in the development and
application of pharmacometric methodology for enhancing drug
treatment, development and regulation. From 1980 to 1984, Bill was a
Medical Research Associate with the Biopharmaceutics Unit of The
Upjohn Company. From 1987 to 1993, he was Assistant Professor of
Pharmaceutics at The University of Texas at Austin, College of
Pharmacy. In 1993 he joined the Center for Drug Evaluation and
Research, U.S. Food and Drug Administration where he served as the
Associate Director for Scientific Affairs, Office of Clinical
Pharmacology and Biopharmaceutics. From 1997 to 1999 he was the Vice
President for Pharmacokinetic Research and Development at GloboMax
LLC. He then joined the Pharsight Corporation as a Senior Scientific
Consultant from 1999 to 2007. His research interests include
theoretical and computer analysis of pharmacokinetic and
pharmacodynamic systems. Recent efforts have concentrated on the use
of Bayesian modeling and simulation to optimize decision-making in
clinical drug development and treatment. He is currently seeking to
make Bayesian modeling and simulation methods more accessible via
training programs and development of software tools.
\\

\noindent
{\sc Relevant Awards}
\\[2pt]
2014 ISoP Innovation Award
\\

\noindent
{\sc Relevant Publications}
\vspace*{-3pt}
\begin{enumerate}
\item Jorge Luiz Gross, James Rogers, Daniel Polhamus, William
  Gillespie, Christian Friedrich, Yan Gong, Brigitta Ursula Monz,
  Sanjay Patel, Alexander Staab, and Silke Retlich. A novel
  model-based meta-analysis to indirectly estimate the comparative
  efficacy of two medications: an example using DPP-4 inhibitors,
  sitagliptin and linagliptin, in treatment of type 2 diabetes
  mellitus. {\it BMJ Open}, 3:e001844
  (http://bmjopen.bmj.com/content/3/3/e001844), 2013. 
\item James A.\ Rogers, Daniel Polhamus, William R.\ Gillespie, Kaori Ito,
  Klaus Romero, Ruolun Qiu, Diane Stephenson, Marc R.\ Gastonguay, and
  Brian Corrigan. Combining patient-level and summary-level data for
  alzheimer’s disease modeling and simulation: a beta regression
  meta-analysis. {\it J Pharmacokinet Pharmacodyn}, 39(5):479--98, 2012
\item Timothy T.\ Bergsma, William Knebel, Jeannine Fisher, William R.\
  Gillespie, Matthew M.\ Riggs, Leonid Gibiansky, and Marc R.\
  Gastonguay. Facilitating pharmacometric workflow with the metrumrg
  package for R. {\it Comput Methods Programs Biomed}, 109(1):77--85, Jan
  2013. 
\item G.\ Stagni, A.\ M.\ Shepherd, Y.\ Liu, and W.\ R.\
  Gillespie. Bioavailability assessment from pharmacologic data:
  method and clinical evaluation. {\it J Pharmacokinet Biopharm},
  25(3):349--362, Jun 1997.
\item G.\ Stagni, A.\ M.\ Shepherd, Y.\ Liu, and W.\ R.\ Gillespie. New
  mathematical implementation of generalized pharmacodynamic models:
  method and clinical evaluation. {\it J Pharmacokinet Biopharm},
  25(3):313--348, Jun 1997. 
\end{enumerate}

\vspace*{12pt}
\noindent
{\sc Marc R. Gastonguay}
\\[2pt]
Georgetown University, Ph.D. Pharmacology, 1994 \\
University of Connecticut, B.S. Pharmacy, 1989
\\

\noindent
{\sc Relevant Experience} \\[2pt]
Gastonguay has more than 20 years experience in applying mathematical
modeling and simulation to problems in clinical pharmacology and drug
development across industry, government, and academic settings. He is
als an entrepreneur, founding two private companies, and a non-profit
organization. He is currently President and CEO of Metrum Research
Group, Scientific Director of Metrum Institute, and a member of the
adjunct faculty at the University of Connecticut, Department of
Biomedical Engineering and the University of Pennsylvania, Perelman
School of Medicine.  
\\  

\noindent
{\sc Relevant Awards} 
\\[2pt]
2011 Innovation Award, American Society of Pharmacometrics
\\
2010--2015 Marcum Tech Top 40 Award, Connecticut Technology Council
\\
2014 Fellow, International Society of Pharmacometrics
\\

\noindent
{\sc Relevant Publications}
\vspace*{-3pt}
\begin{enumerate}
\item Flockhart, D., Bies, R.R., Gastonguay, M.R., Schwartz, S.L. Big
  Data: Challenges and Opportunities for Clinical Pharmacology.  {\it
    Br J Clin Pharmacol}, Feb 2016 [Epub ahead of print].
\item Eudy, R.J., Gastonguay, M.R., Baron, K.T., Riggs,
  M.M. Connecting the Dots: Linking Osteocyte Activity and Therapeutic
  Modulation of Sclerostin by Extending a Multiscale Systems
  Model. {\it CPT Pharmacometrics Syst Pharmacol}, 4(9):527-36, Sep
  2015 [Epub].
\item Rosario, M., Dirks, N.L., Gastonguay, M.R., Fasanmade, A.A.,
  Wyant, T., Parikh, A., Sandborn, W.J., Feagan, B.G., Reinisch,
  W. Fox, I. Population pharmacokinetics-pharmacodynamics of
  vedolizumab in patients with ulcerative colitis and Crohn's
  disease. {\it Aliment Pharmacol Ther}, 42(2):188--202, Jul 2015.
\item Knebel, W., Rogers, J., Polhamus, D., Ermer, J., Gastonguay,
  M.R. Modeling and simulation of the exposure-response and dropout
  pattern of guanfacine extended-release in pediatric patients with
  ADHD. {\it J Pharmacokinet Pharmacodyn}, 42(1):45--65, Feb 2015.
\item Ravva, P., Gastonguay,M.R., Faessel, H.M., Lee, T.C., Niaura,
  R. Pharmacokinetic-pharmacodynamic modeling of the effect of
  varenicline on nicotine craving in adult smokers. {\it Nicotine Tob
    Res}, 17(1):106--13, Jan 2015.
\item Jin, Y., Bies, R., Gastonguay, M.R., Wang, Y., Stockbridge, N.,
  Gobburu, J., Madabushi, R. Predicted impact of various clinical
  practice strategies on cardiovascular risk for the treatment of
  hypertension: a clinical trial simulation study. {\it J
    Pharmacokinet Pharmacodyn},41(6):693-704. Dec 2014.
\end{enumerate}

\subsection{Columbia University}

{\sc Andrew Gelman}
\\[2pt]
Massachusetts Institute of Technology, B.S.\ Mathematics, 1985 \\
Massachusetts Institute of Technology, B.S.\ Physics, 1986 \\
Harvard University, Ph.D.\ Statistics, 1990
\\

\noindent
{\sc Relevant Experience}
\\[2pt]
Gelman has decades of experience in applying Bayesian data analysis to
scientific applications, and has developed many widely-used methods
for Bayesian inference, computation, and model checking. Examples
include the standard method for monitoring the convergence of
iterative simulations (Gelman and Rubin, 1992), a seminal paper on
posterior predictive checking (Gelman, Meng, and Stern, 1996), path
sampling (Gelman and Meng, 1998), prior distributions for hierarchical
models and logistic regression (Gelman, 2006, Gelman, Jakulin, et al.,
2008). Gelman is also one of the creators of Stan, the Bayesian
software that is central to this proposal.
\\

\noindent
{\sc Relevant Awards}
\\[2pt]
2012 Open Source Software World Challenge award for Stan: An R and C++
package for Bayesian sampling
\\
1998 Outstanding Statistical Application award from the American
Statistical Association for ``Physiological pharmacokinetic analysis
using population modeling and informative prior distributions.''
\\

\noindent
{\sc Relevant Publications}
\vspace*{-3pt}
\begin{enumerate}
\item Kucukelbir, A., Ranganath, R., Gelman, A., and Blei,
  D. (2015). Automatic variational inference in stan. In {\it Advances in
  Neural Information Processing Systems} (pp. 568--576).
\item Gelman, A., Lee, D., and Guo, J. (2015). Stan A Probabilistic
  Programming Language for Bayesian Inference and Optimization. {\it
    Journal of Educational and Behavioral Statistics},
  1076998615606113.
\item Bob Carpenter, Andrew Gelman, Matt Hoffman, Daniel Lee, Ben
  Goodrich, Michael Betancourt, Marcus Brubaker, Jiqiang Guo, Peter
  Li, and Allen Riddell. (2016) Stan: A probabilistic programming
  language.  {\it Journal of Statistical Software} in press.
\item Hoffman, M. D., and Gelman, A. (2014). The no-U-turn sampler:
  Adaptively setting path lengths in Hamiltonian Monte Carlo. {\it The
    Journal of Machine Learning Research}, 15(1), 1593--1623.
\item Gelman, A., Jakulin, A., Pittau, M. G., and Su, Y. S. (2008). A
  weakly informative default prior distribution for logistic and other
  regression models. {\it The Annals of Applied Statistics}, 1360-1383.
\item Gelman, A. (2006). Prior distributions for variance parameters
  in hierarchical models (comment on article by Browne and
  Draper). {\it Bayesian Analysis}, 1(3), 515--534.
\item Gelman, A., and Meng, X. L. (1998). Simulating normalizing
  constants: From importance sampling to bridge sampling to path
  sampling. {\it Statistical Science}, 163--185.
\item Gelman, A., Bois, F., and Jiang, J. (1996). Physiological
  pharmacokinetic analysis using population modeling and informative
  prior distributions. {\it Journal of the American Statistical
    Association}, 91(436), 1400--1412.
\item Bois, F. Y., Gelman, A., Jiang, J., Maszle, D. R., Zeise, L.,
  and Alexeef, G. (1996). Population toxicokinetics of
  tetrachloroethylene. {\it Archives of Toxicology}, 70(6), 347--355.
\item Gelman, A., Meng, X. L., and Stern, H. (1996). Posterior predictive
assessment of model fitness via realized discrepancies. {\it Statistica
Sinica}, 733--760.
\item Gelman, A., and Rubin, D. B. (1992). Inference from iterative
  simulation using multiple sequences. {\it Statistical Science},
  457--472.
\end{enumerate}

\vspace*{12pt}

\noindent
{\sc Bob Carpenter}
\\[2pt]
Michigan State University, B.S.\ Mathematics, 1984
\\
University of Edinburgh, Ph.D.\ Cognitive and Computer Science, 1989
\\

\noindent
{\sc Relevant Experience}
\\[3pt]
Carpenter has decades of experience in programming language design and
open-source software development.  He was an associate professor of
computational linguistics at Carnegie-Mellon University, a research
scientist at Bell Laboratories (Lucent), then worked as a production
programmer at SpeechWorks and Alias-i.  Carpenter designed the
programming language component of Stan and is its principal developer.
His first open-source package (ALE, for natural language grammar
development) implements the theory from his first book (Carpenter
1992) and is still in use in natural language processing and
linguistics.  His second open-source package (LingPipe; for
statistical natural language processing) is widely deployed in
production and is cited in over 1000 publications.  He is now a
research scientist in the department of statistics at Columbia
University, working full time on Stan.
\\

\noindent
{\sc Relevant Awards}
\\[2pt]
2012 Open Source Software World Challenge award for Stan: An R and C++
package for Bayesian sampling

\vspace*{12pt}

\noindent
{\sc Relevant Publications}
\vspace*{-3pt}
\begin{itemize}
\item B.\ Carpenter, A.\ Gelman, M.\ Hoffman, D.\ Lee, B.\ Goodrich,
  M.\ Betancourt, M.\ Brubaker, J.\ Guo, P.\ Li, A.\ Riddell. 2016.
  Stan: A Probabilistic Programming Language. {\it Journal of
    Statistical Software}.  In press.
\item D. Lee, B. Carpenter, P. Li, M. Betancourt, A. Gelman. 2014. Stan: A
  Platform for Bayesian Inference.  Paper presented at the {\it 3rd
    NIPS Workshop on Probabilistic Programming}.
\item S. Weber, B. Carpenter, D. Lee, F. Bois, A. Gelman,
  A. Racine. 2014. Bayesian Drug Disease Model with Stan---Using published
  longitudinal data summaries in population models. {\it
    Abstracts of the Annual Meeting of the Population Group in
    Europe}.
\item D. Lee, B. Carpenter, P. Li, M. Betancourt, A. Gelman. 2014. Stan: A
  Platform for Bayesian Inference. {\it 3rd NIPS Workshop on
    Probabilistic Programming}.
\item B.~Carpenter, M.~Hoffman and A.~Gelman. 2012. Stan: Scalable
  software for probabilistic modeling.  {\it 1st NIPS Workshop on
    Probabilistic Programming}.
\item B.~Carpenter. 1992.  {\it The Logic of Typed Feature Structures:
    With Applications to Unification Grammars, Logic Programs, and
    Constraint Resolution}.  Cambridge Tracts in Theoretical Computer
  Science. Cambridge University Press.
\end{itemize}

\noindent
{\sc Daniel Lee}
\\[2pt]
University of Cambridge, MASt Statistics, 2009
\\
Massachusetts Institute of Technology, B.~S.\ Mathematics with Computer Science, 2004
\\

\noindent
{\sc Relevant Experience}
\\[2pt]
Daniel is a researcher at Columbia University and has been a core
developer of Stan since its start in 2011. He has been applying
Bayesian methods to problems including PK/PD models. Prior to Columbia,
he worked as a software engineer for defense contractors in San Diego.
\\

\noindent
{\sc Relevant Awards}
\\[2pt]
2012 Open Source Software World Challenge award
for Stan: An R and C++ package for Bayesian sampling \\

\noindent
{\sc Relevant Publications}
\vspace*{-3pt}
\begin{itemize}
\item B.\ Carpenter, A.\ Gelman, M.\ Hoffman, D.\ Lee, B.\ Goodrich,
  M.\ Betancourt, M.\ Brubaker, J.\ Guo, P.\ Li, A.\ Riddell. 2016.
  Stan: A Probabilistic Programming Language. {\it Journal of
    Statistical Software}.  In press.
\item A.\ Gelman, D.\ Lee, and J.\ Guo. Stan: A probabilistic
  programming language for Bayesian inference and
  optimization. 2015. {\it Journal of Educational and Behavioral
    Statistics}.
\item D. Lee, B. Carpenter, P. Li, M. Betancourt, A. Gelman. 2014. Stan: A
  Platform for Bayesian Inference. {\it 3rd NIPS Workshop on
    Probabilistic Programming}.
\item S. Weber, B. Carpenter, D. Lee, F. Bois, A. Gelman,
  A. Racine. 2014. Bayesian Drug Disease Model with Stan---Using published
  longitudinal data summaries in population models. {\it
    Abstracts of the Annual Meeting of the Population Group in
    Europe}.
\end{itemize}

% Michael Betancourt and Eric Novik are not at Columbia


\section{Foreign Citizens.}

% Identify any foreign citizens or individuals holding dual
% citizenship expected to be involved on this project as a direct
% employee, subcontractor, or consultant. For these individuals,
% please specify their country of origin, the type of visa or work
% permit under which they are performing and an explanation of their
% anticipated level of involvement on this project. You may be asked
% to provide additional information during negotiations in order to
% verify the foreign citizen’s eligibility to participate on a SBIR
% contract. Supplemental information provided in response to this
% paragraph will be protected in accordance with the Privacy Act (5
% U.S.C. 552a), if applicable, and the Freedom of Information Act (5
% U.S.C. 552(b)(6)).

None at either Metrum Research Group or Columbia University.

\section{Facilities/Equipment.}

% 1/2 page
% (Marc for Metrum)

% Describe available instrumentation and physical facilities necessary
% to carry out the Phase I effort. Justify equipment purchases in this
% section and include detailed pricing information in the cost volume.
% State whether or not the facilities where the proposed work will be
% performed meet environmental laws and regulations of federal, state
% (name), and local Governments for, but not limited to, the following
% groupings: airborne emissions, waterborne effluents, external
% radiation levels, outdoor noise, solid and bulk waste disposal
% practices, and handling and storage of toxic and hazardous
% materials. (half page)

\subsection{Metrum Research Group}

Since the primary focus of this project is computer software and
development, the equipment requirements are primarily suitable
computer hardware and software resources. All Metrum scientists are
equipped with a recent model Apple MacBook Pro. Most production
computation is done on Amazon Web Services (AWS) cloud compute servers
using the Ubuntu operating system. The software required for the
proposed project is installed on the standard machine image
configuration used by Metrum, e.g., Gnu C++ compiler, R, Gnu Fortran,
NONMEM, LaTeX, Stan, git and subversion version control software.

\subsection{Columbia University}

At Columbia University, all participants are equipped with recent
Apple Macbook Pro personal computers.  The Stan project has three
dedicated servers with 64GB and 8 cores each, one for each of Mac OS
X, Linux, and Windows, along with two dedicated workstations for web
hosting and remote debugging on Windows and Linux.  The Stan team has
access to two Linux clusters, one shared by statistics and astronomy
with 384 cores, and one shared by 24 departments with 2672 CPU cores,
35 of which are high end 256-gigabyte memory nodes, and 45,000+ NVIDIA
K40 and K20 GPU cores.  All of the software required for individual
computers for this project (see the list above) is free and open
source.  Source version control, issue trackers, and project web
servers are provided by GitHub, mailing lists for user and developer
groups is supplied by Google, and continuous integration for the
interfaces by Travis;  all of these services are free to open-source
projects.


\section{Subcontractors/Consultants.}

% 1/2 page

% Involvement of a university or other subcontractors or consultants
% in the project may be appropriate. If such involvement is intended,
% it should be identified and described according to the Cost
% Breakdown Guidance. A minimum of

% 40\% of the research and/or analytical work in Phase I, as measured
% by direct and indirect costs, must be carried out by the proposing
% firm, and a minimum of 30\% of the research and/or analytical work
% in Phase I, as measured by direct and indirect costs, must be
% carried out by the Research Partner. Eligible STTR Research partners
% include subcontracts with Universities and Federally Funded Research
% and Development Centers (FFRDCs). A waiver is no longer required for
% the use of federal laboratories but they do not qualify as a
% Research Partner; proposers may only subcontract to federal
% laboratories within the remaining 30\% and must certify their use of
% such facilities on the Cover Sheet of the proposal. Subcontracts
% with other federal organizations are not permitted. (half page)

Approximately 50\% of the project work will be conducted by the Stan
development team based in Columbia University. The key personnel in
that group are listed in section \ref{personnel}.

\section{Prior, Current or Pending Support of Similar Proposals or Awards.}

% Metrum Research Group has no prior, current or pending support for a

% similar proposal.

% If a proposal submitted in response to this solicitation is
% substantially the same as another proposal that was funded, is now
% being funded, or is pending with another Federal Agency or another
% or the same DoD Component, you must reveal this on the Proposal
% Cover Sheet and provide the following information (half page):

% a) Name and address of the Federal Agency(s) or DoD Component to
% which a proposal was submitted, will be submitted, or from which an
% award is expected or has been received.

% b) Date of proposal submission or date of award.

% c) Title of proposal.

% d) Name and title of principal investigator for each proposal
% submitted or award received.

% e) Title, number, and date of solicitation(s) under which the
% proposal was submitted, will be submitted, or under which award is
% expected or has been received.

% f) If award was received, state contract number.

% g) Specify the applicable topics for each SBIR proposal submitted or
% award received.

% Note: If this does not apply, state in the proposal "No prior,
% current, or pending support for proposed work."

No prior, current, or pending support for proposed work.

% this header implicit from bibliography:  \section*{References}

\nocite{stan-development-team:2016}

\makeatletter
\makeatother
\bibliographystyle{abbrv}
\bibliography{onr}

\end{document}

