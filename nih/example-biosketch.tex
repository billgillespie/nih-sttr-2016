% This example file for NIH biographical sketches was originally written
% by Bruce Donald (http://www.cs.duke.edu/brd/).
% 
% You may freely use, modify and/or distribute this file.
% 
\documentclass[11pt]{nih}
%\documentclass[12pt]{nih-times}
% last revision:
\def\mydate{2006-09-16 17:10:33 karl}
%\usepackage[dvips]{graphicx}
%\usepackage{psfrag}
\usepackage{color}
%\usepackage{boxedminipage}
\usepackage{amsfonts}
\usepackage{denselists}

\long\def\gobble#1{}

\long\def\MEMS#1{}

%Note from brd
\long\def\todo#1{}
\def\ICRA{IEEE International Conference on Robotics and Automation (ICRA)}

\def\cbk#1{[{\em #1}]}

\def\degree{$^\circ$}
\def\R{\mathbb{R}}
\def\Fscr{\mathcal{F}}
\def\set#1{{\{#1\}}}
\def\edge{\!\rightarrow\!}
\def\dedge{\!\leftrightarrow\!}

\def\Jigsaw{{\sc Jigsaw}}
\def\ahelix{\ensuremath{\alpha}-helix}
\def\ahelices{\ensuremath{\alpha}-helices}
\def\ahelical{$\alpha$-helical}
\def\bstrand{\ensuremath{\beta}-strand}
\def\bstrands{\ensuremath{\beta}-strands}
\def\bsheet{\ensuremath{\beta}-sheet}
\def\bsheets{\ensuremath{\beta}-sheets}
\def\hone{\ensuremath{^1}\rm{H}}
\def\htwo{$^{2}$H}
\def\cthir{\ensuremath{^{13}}\rm{C}}
\def\nfif{\ensuremath{^{15}}\rm{N}}
\def\hn{\rm{H}\ensuremath{^\mathrm{N}}}
\def\hnone{\textup{H}\ensuremath{^1_\mathrm{N}}}
\def\ca{\rm{C}\ensuremath{^\alpha}}
\def\catwel{\ensuremath{^{12}}\rm{C}\ensuremath{^\alpha}}
\def\ha{\rm{H}\ensuremath{^\alpha}}
\def\cb{\rm{C}\ensuremath{^\beta}}
\def\hb{\rm{H}\ensuremath{^\beta}}
\def\hg{\rm{H}\ensuremath{^\gamma}}
\def\dnn{\ensuremath{d_{\mathrm{NN}}}}
\def\dan{\ensuremath{d_{\alpha \mathrm{N}}}}
\def\jconst{\ensuremath{^{3}J_{\mathrm{H}^{\mathrm{N}}\mathrm{H}^{\alpha}}}}

\def\cbfb{CBF-$\beta$}

\newtheorem{defn}{Definition}
\newtheorem{claim}{Claim}

\newenvironment{closeenumerate}{\begin{list}{\arabic{enumi}.}{\topsep=0in\itemsep=0in\parsep=0in\usecounter{enumi}}}{\end{list}}
\def\CR{\hspace{0pt}}           % ``invisible'' space for line break

%\long\def\efforta mount#1{#1}
\long\def\effortamount#1{}

\begin{document}

%\setcounter{page}{20} % or whatever
\setcounter{page}{7} % or whatever

%\newsec{Biographical Sketch: Bruce Randall Donald}
\def\vp{\vphantom{\Large{O}}}
\def\Vp{\vphantom{\LARGE{O}}}


\subsubsection*{Biographical Sketch}

%\gobble{

\begin{tabular}{|l|l|}
\hline
\Vp{\sf Name} & {\sf Position Title}\\
\hline
\Vp \ \ {\bf Bruce R.~Donald, Ph.D.} & \ 
 {\bf   Professor of Computer Science and Biochemistry}\\
\gobble{{\bf   William and Sue Gross Professor  }\\
 & \ \ \ \ \ \ {\bf  of Computer Science and Biochemistry (as of 8/1/2006)}\\}
\hline
\end{tabular}

%}

\subsubsection*{Education/Training}

\begin{tabular}{|l|c|c|l|}
\hline
\Vp{\sf INSTITUTION} & 	{\sf DEGREE} &
     {\sf YEAR} & {\sf	FIELD OF STUDY}\\
\hline
\hline
\Vp Yale University	& B.A.	& 1980	& \\
\hline
\Vp Massachusetts Institute of Technology	& S.M.&	1984 &	Electrical Engineering \& Computer Science\\
\hline
\Vp Massachusetts Institute of Technology &	Ph.D.	&1987&
Computer Science\\
\hline
\end{tabular}

\bigskip

\gobble{
\noindent
\begin{Description}
\item[]
Ph.D.~Advisor: Tom\'as Lozano-P\'erez, MIT Artificial Intelligence
Laboratory and Department of Electrical Engineering \& Computer Science.
\end{Description}
}

\subsubsection*{A. Research and Professional Experience}

\underline{Professional Experience}

\def\myitem{\ $\bullet$\ }

 %\def\parens#1{({#1})}
\def\parens#1{{#1}:}

%\begin{Description}
{\myitem} \parens{1978-84} Research Analyst,
Laboratory for Computer Graphics and Spatial
Analysis, GSD, Harvard University.  
{\myitem} \parens{1984} Research Staff, Artificial Intelligence Laboratory, MIT.
{\myitem} \parens{1982-1987} Graduate student, Artificial Intelligence Laboratory and Department of EECS, MIT. 
{\myitem} \parens{1987-1993} Assistant Professor; 
\parens{1993-1998} Associate Professor (with tenure), Computer Science Department, Cornell University.
{\myitem} \parens{1995-1996} Consultant and Contractor, Interval Research Corporation, Palo Alto, CA.
{\myitem} \parens{1994-1996} Visiting Professor, Computer Science Department,
Stanford. 
{\myitem} \parens{1997-1999} Associate Professor (with tenure);
  \parens{1999-2006} Professor of Computer Science,
  Dartmouth College.
\gobble{{\myitem} \parens{1998} Founder, M.D.-Ph.D.~Program in Computational Biology, Dartmouth College and
Dartmouth Medical School.}
{\myitem} \parens{1999-2006} M.D.-Ph.D.~Committee, Dartmouth College and
Dartmouth Medical School.
{\myitem} \parens{2000} Conference chair, Int'l.~Workshop on Algorithmic
Foundations of Robotics.
{\myitem} 11 NSF Advisory Panels (1990-2003).
{\myitem} \parens{2000-2001} Visiting Scientist, Artificial Intelligence Laboratory and Department of EECS, MIT.
{\myitem} \parens{2000-2002} Scientific Advisory Board, Carta
Proteomics, Inc. (now ExSAR). 
{\myitem} \parens{2000-2006} Member, Dartmouth Center for Structural
Biology and Computational Chemistry.
{\myitem} \parens{2000-2006} Adjunct Professor of Chemistry, Dartmouth.
{\myitem} \parens{2002-2006} Adjunct Professor of Biological Sciences, Dartmouth.
{\myitem} \parens{2002, 2004, 2005} {\em Ad hoc} Member, NIH Study Sections (BBCA,
 ALY, ZRG1 CFS).
{\myitem}  \parens{2003--2006} Joan and Edward Foley Professor, Dartmouth.
{\myitem}\parens{2006--now} William and Sue Gross Professor of Computer
Science, Duke University. 
{\myitem}\parens{2006--now}
Professor of Biochemistry, School of Medicine, Duke University Medical Center.

%\end{Description}

\noindent\underline{Honors and Awards}

%\begin{Description}

{\myitem} \parens{1979} Phi Beta Kappa; \parens{1980} Distinction in the Major, Yale University.
{\myitem} \parens{1980} Graduated  {\em Summa Cum Laude,} Yale.
{\myitem} \parens{1985-1987} NASA/JPL Graduate Student Researcher Fellowship.
{\myitem} \parens{1989-1994} NSF Presidential Young Investigator.
{\myitem} \parens{1997-2000} NSF Challenges in Computer and Information Science and Engineering Grant.	
{\myitem} \parens{2001} Guggenheim Fellow, {\em
``Algorithms in Structural Proteomics."}
{\myitem} \parens{2002} Distinguished Lectures, Robert Mueller-Thuns
(Univ.~Illinois, Urbana-Champaign); Triangle (UNC Chapel Hill, Duke
\& N.C.~State).

%\end{Description}

\def\mystar{{$\star$}}

%\subsubsection*{Selected Peer-reviewed Publications (from a list of
%159 total, with 28 in 2003-4). }

\medskip

\noindent{\bf B. Selected Peer-reviewed Publications (from a list of
176 total, with 19 in 2005--6)}\gobble{\\ {\sf Publications available online at
  {\tt
  www.cs.dartmouth.edu/brd/Research/Bio/}} }

\vspace*{-0.1in}

\gobble{{{\mystar} {\em Note: In Computer Science, certain conferences
(Marked with a $\star$) are are highly selective and rigorously
refereed, often by 3 reviewers plus the conference chairs. Conference
papers are published not as one-page abstracts, but as 8-12 page full
papers (in 10pt double-column format). For this reason, conference
papers are considered primary publications in the field.  Selectivity
can be roughly quantitated by the acceptance rate.}}}

\def\Nospacing{\itemsep=0pt\topsep=0pt\partopsep=0pt\parskip=0pt\parsep=0pt}

\makeatletter
\def\thebibliography#1{\list
{[\arabic{enumiv}]}{\settowidth\labelwidth{[#1]}\leftmargin\labelwidth
\advance\leftmargin\labelsep
\usecounter{enumiv}\Nospacing}
\def\newblock{\hskip .11em plus .33em minus .07em}
\sloppy\clubpenalty4000\widowpenalty4000
\sfcode`\.=1000\relax
}
\let\endthebibliography=\endlist
\makeatother

%%%%%%% begin papers/biblio
%\begin{thebibliography}{10}

\def\th{$^{\rm{th}}$}

%\def\mybibitem#1#2{\bibitem{#1}{#2}}
\def\brd{B.~R. Donald}
%\def\brd{{\bf B.~R. Donald}}

\def\mybibitem#1{\item}
\begin{Enumerate}

%\input{papers-r01}
%% %\begin{thebibliography}{10}

%% \def\th{$^{\rm{th}}$}

%% %\def\mybibitem#1#2{\bibitem{#1}{#2}}

%% \def\mybibitem#1{\item}
%% \begin{Enumerate}

\bibitem{KapurMundyDonald92}
B.~R. Donald, D.~Kapur, and J.~Mundy.
\newblock {\em Symbolic and Numerical Computation for Artificial Intelligence}.
\newblock Academic Press, Harcourt Jovanovich, London, 1992.

\bibitem{CannyDonaldReifXavier93}
B.~R. Donald, P.~Xavier, J.~Canny, and J.~Reif.
\newblock Kinodynamic motion planning.
\newblock {\em Journal of the ACM}, 40(5):1048--1066, 1993.

\bibitem{jcb00-jigsaw}
C.~Bailey-Kellogg, A.~Widge, J.~J. {Kelley III}, M.~J. Berardi, J.~H.
  Bushweller, and {\brd}.
\newblock The {NOESY} {Jigsaw}: Automated protein secondary structure and
  main-chain assignment from sparse, unassigned {NMR} data.
\newblock {\em Jour. Comp. Biol.}, 3-4(7):537--558, 2000.

\bibitem{jcb00-sar}
C.~Bailey-Kellogg, J.~J. {Kelley III}, C.~Stein, and {\brd}.
\newblock Reducing mass degeneracy in {SAR} by {MS} by stable isotopic
  labeling.
\newblock {\em Jour. Comp. Biol.}, 8(1):19--36, 2001.

\gobble{
\bibitem{icra01}
C.~Bailey-Kellogg, J.~J. {Kelley III}, R.~Lilien, and {\brd}.
\newblock Physical geometric algorithms for structural molecular biology.
\newblock In {\em the Special Session on Computational Biology \& Chemistry,
  {\it Proc.~{IEEE} Int'l Conf. on Robotics and Automation
  ({ICRA})}}, pp.~940--947, May 2001.
}

\bibitem{recomb01}
C.~Langmead and {\brd}.
\newblock Extracting structural information using time-frequency analysis of
  protein {NMR} data.
\newblock In {\em Proc.~5{\th}  Int'l.~Conf.~on
  Research in Computational Molecular Biology (RECOMB)}, pp.~164--175. ACM
  Press, April 2001.

\gobble{
\bibitem{csb02}
C.~Langmead, C.~R. McClung, and {\brd}.
\newblock A maximum entropy algorithm for rhythmic analysis of genome-wide
  expression patterns.
\newblock In {\em Proc.~IEEE Computer Society Bioinformatics
  Conference (IEEE CSB)}, pp.~237--245, August 2002.
}

\bibitem{jcb-rage}
C.~Langmead, A.~Yan, C.~R. McClung, and {\brd}.
\newblock Phase-independent rhythmic analysis of genome-wide expression
  patterns.
\newblock {\em Journal of Computational Biology}, 10(3-4):521--536, 2003.

\bibitem{jcb-fld03}
R.~Lilien, H.~Farid, and {\brd}.
\newblock Probabilistic disease classification of expression-dependent
  proteomic data from mass spectrometry of human serum.
\newblock {\em Journal of Computational Biology}, 10(6):925--946, 2003.

\bibitem{recomb03}
C.~Langmead, A.~Yan, R.~Lilien, L.~Wang, and {\brd}.
\newblock A polynomial-time nuclear vector replacement algorithm for automated
  {NMR} resonance assignments.
\newblock In {\em Proc.~7{\th} Int'l.~Conf.~on
  Research in Computational Molecuar Biology (RECOMB)}, pp.~176--187,
  Berlin, Germany, April 2003. ACM Press.

\bibitem{ieeecsb-langmead03}
C.~Langmead and {\brd}.
\newblock 3{D} structural homology detection via unassigned residual dipolar
  couplings.
\newblock In {\em Proc.~{IEEE} Computer Society Bioinformatics
  Conference ({CSB})}, pp.~209--217, Stanford, Aug.~2003.

\bibitem{oneil-jbc03}
R.~O'Neil, R.~Lilien, {\brd}, R.~Stroud, and A.~Anderson.
\newblock Phylogenetic classification of protozoa based on the structure of the
  linker domain in the bifunctional enzyme, dihydrofolate reductase-thymidylate
  synthase.
\newblock {\em Jour. Biol. Chem.}, 278(52):52980--52987, 2003.

\bibitem{oneil-jem03}
R.~O'Neil, R.~Lilien, {\brd}, R.~Stroud, and A.~Anderson.
\newblock The crystal structure of dihydrofolate reductase-thymidylate synthase
  from {{\em Cryptosporidium hominis}} reveals a novel architecture for the
  bifunctional enzyme.
\newblock {\em Jour. Eukaryotic Microbiology}, 50(6):555--556, 2003.

\gobble{
\bibitem{ieeecsb03-wang}
L.~Wang, R.~Mettu, R.~Lilien, and {\brd}.
\newblock An exact algorithm for determining protein backbone structure from
  {NH} residual dipolar couplings.
\newblock In {\em Proc.~{IEEE} Computer Society Bioinformatics
  Conference ({CSB})}, pp.~611--612, Stanford, August 2003.
}

\bibitem{langmead-jbnmr04}
C.~Langmead and {\brd}.
\newblock An expectation/maximization nuclear vector replacement algorithm for
  automated {NMR} resonance assignments.
\newblock {\em Jour. Biomolecular {NMR}}, 29(2):111--138, 2004.

\bibitem{wang-jbnmr03}
L.~Wang and {\brd}.
\newblock Exact solutions for internuclear vectors and backbone dihedral angles
  from {NH} residual dipolar couplings in two media, and their application in a
  systematic search algorithm for determining protein backbone structure.
\newblock {\em Jour. Biomolecular {NMR}}, 29(3):223--242, 2004.

\bibitem{langmead-jcb04}
C.~Langmead, A.~Yan, R.~Lilien, L.~Wang, and {\brd}.
\newblock A polynomial-time nuclear vector replacement algorithm for automated
  {NMR} resonance assignments.
\newblock {\em Jour. Comp. Biol.}, 11(2-3):277--298, 2004.

\bibitem{recomb-04}
R.~Lilien, B.~Stevens, A.~Anderson, and {\brd}.
\newblock A novel ensemble-based scoring and search algorithm for protein
  redesign, and its application to modify the substrate specificity of the
  gramicidin synthetase {A} phenylalanine adenylation enzyme.
\newblock In {\em Proc.~Eighth Annual International Conference on
  Research in Computational Molecular Biology ({RECOMB})}, pp.~46--57, San
  Diego, March 2004.

\bibitem{lilien-acd04}
R.~Lilien, C.~Bailey-Kellogg, A.~Anderson, and {\brd}.
\newblock {A subgroup algorithm to identify cross-rotation peaks consistent
  with non-crystallographic symmetry}.
\newblock {\em Acta Crystallographica Section D: Biological Crystallography},
  60(6):1057--1067, Jun 2004.

\bibitem{LangmeadDonald-csb04}
C.~Langmead and {\brd}.
\newblock High-throughput 3{D} structural homology detection via {NMR}
  resonance assignment.
\newblock In {\em Proc.~{IEEE} Computational Systems
  Bioinformatics Conference ({CSB})}, pp.~278--289, Stanford, CA, August
  2004.

\gobble{
\bibitem{Dartmouth:TR2004-492}
R.~Lilien, M.~Sridharan, and {\brd}.
\newblock {Identification of Novel Small Molecule Inhibitors of Core-Binding
  Factor Dimerization by Computational Screening against NMR Molecular
  Ensembles}.
\newblock Technical Report TR2004-492, Dartmouth College, Computer Science,
  Hanover, NH, March 2004.
}

\bibitem{WangDonald-csb04}
L.~Wang and {\brd}.
\newblock Analysis of a systematic search-based algorithm for determining
  protein backbone structure from a minimal number of residual dipolar
  couplings.
\newblock In {\em Proc.~{IEEE} Computational Systems
  Bioinformatics Conference ({CSB})}, pp.~319--330, Stanford, CA, August
  2004.

\bibitem{ismb05}
R.~Mettu, R.~Lilien, and B.~R. Donald.
\newblock High-throughput inference of protein-protein interfaces from
  unassigned {NMR} data.
\newblock {\em Bioinformatics}, 2005; {\bf{21}}(Suppl.~1):i292--i301.   
\gobble{special issue from papers presented at the 2005 Int'l.~Conf.~on
Intelligent Systems for Molecular Biology, Detroit, MI) }

\mybibitem{jcb04-NRPS}
R.~Lilien, B.~Stevens, A.~Anderson, and {\brd}.
\newblock A novel ensemble-based scoring and search algorithm for protein
  redesign, and its application to modify the substrate specificity of the
  gramicidin synthetase {A} phenylalanine adenylation enzyme.
\newblock {\em Journal of Computational Biology} 2005; {\bf{12}}(6-7):740--761.

\mybibitem{csb05-noe}
L.~Wang and B.~R. Donald.
\newblock An efficient and accurate algorithm for assigning nuclear
  {Overhauser} effect restraints using a rotamer library ensemble and residual
  dipolar couplings.
\newblock In {\em Proceedings of the {IEEE} Computational Systems
  Bioinformatics Conference ({CSB})}, pp.~189--202,
Stanford, CA, August 2005.

\mybibitem{csb05-poly}
L.~Wang, R.~Mettu, and B.~R. Donald.
\newblock An algebraic geometry approach to protein backbone structure
  determination from {NMR} data.
\newblock In {\em Proceedings of the {IEEE} Computational Systems
  Bioinformatics Conference ({CSB})}, pp.~235--246, Stanford, CA,
August 2005.

\gobble{
\mybibitem{recomb-06}
I.~Georgiev, R.~Lilien, and B.~R. Donald.
\newblock A novel minimized dead-end elimination criterion and its application
  to protein redesign in a hybrid scoring and search algorithm for computing
  partition functions over molecular ensembles.
\newblock In {\em Proc.~Tenth Ann.~Intl.~Conf.~on
  Research in Computational Molecular Biology (RECOMB)}, pp. 530--545,
  Venice, Italy, April 2006. Springer Berlin, Lecture Notes in
  Computer Science, LNBI 3909.
}

\bibitem{ismb06}
I.~Georgiev, R.~Lilien, and B.~R. Donald.
\newblock Improved pruning algorithms and divide-and-conquer strategies for
  dead-end elimination, with application to protein design.
\newblock {\em Bioinformatics} 2006; {\bf{22}}(14):e174--183.
\newblock Special issue on papers from the Int'l Conf. on Intelligent
  Sys. for Mol. Biol. ({ISMB 2006}), Fortaleza, Brazil.

\bibitem{csb06}
L.~Wang and B.~R. Donald.
\newblock A data-driven, systematic search algorithm for structure
  determination of denatured or disordered proteins.
\newblock In {\em Proceedings of the LSS Computational Systems Bioinformatics
  Conference ({CSB})}, Stanford, CA, August 2006.
 Pages 67-78. ISBN 1-86094-700-X.

\bibitem{jcb-poly06}
L.~Wang, R.~Mettu, and B.~R. Donald.
\newblock A polynomial-time algorithm for {\em de novo} protein backbone
  structure determination from {NMR} data.
\newblock {\em Journal of Computational Biology}, 2006.
\newblock In press.

\bibitem{proteins06}
S.~Potluri, A.~Yan, B.~R. Donald, and C.~Bailey-Kellogg.
\newblock Structure determination of symmetric homo-oligomers by a complete
  search of symmetry configuration space using {NMR} restraints and van der
  {Waals} packing.
\newblock {\em Proteins: Structure, Function and Bioinformatics},
  2006; 65(1):203--219.

%\end{Enumerate}

%\end{thebibliography}

\end{Enumerate}

%\end{thebibliography}
%%%%%%% endpapers/biblio


%\input{papers}
%\input{papers2}

\MEMS{Change URL}

\gobble{
\begin{center}
{\sf Publications available online at {\tt www.cs.dartmouth.edu/\lower1.50ex\hbox{\LARGE{\~{}}}brd/Research/Bio/}}
\end{center}
}

%\newpage
%\setcounter{page}{1} % or whatever

\subsection*{C. Research Projects and Funding}

\subsection*{CURRENT SUPPORT}

\def\myhrule{\smallskip\hrule\smallskip}

\myhrule 
\noindent \begin{tabular}{ll}
DONALD, BRUCE R.	&		ONGOING\\
	(PI: Donald)	&	2002-2007	\\
	NIH/NIGMS & R01 GM-65982\\
{\em Automated NMR Assignment and Protein Structure}\\
\effortamount{\$150,000 &\\}
\end{tabular}

The long-term objective of this project is the development of new
computational methods for biomolecular NMR, to be applied in
structural genomics. Two main foci are novel algorithms for automated
assignments, and algorithms for automated structure determination from
solution-state protein NMR.

\myhrule 
\noindent \begin{tabular}{ll}
DONALD, BRUCE R.&			ONGOING \\
(PI: A. Anderson,     Co-investigator: B. Donald) & 2003-2008\\
NIH (NIGMS \& NIAID) &
R01 GM-067542   \\
{\em Design of C.~parvum and T.~gondii DHFR-TS Inhibitors}\\
\effortamount{\$150,000 {(total); Donald sub only: \$11,000.} & \\}
\end{tabular}

The major goal of this project is to design selective and potent
inhibitors against the dihydrofolate reductase (DHFR) domain of
dihydrofolate reductase-thymidylate synthase (DHFR-TS) from
{\em Cryptosporidium hominis} and {\em Toxoplasma gondii.  }

\myhrule 

\subsection*{Completed Research Support}
 
\myhrule
\noindent
\begin{tabular}{ll}
DONALD, BRUCE R.	&		COMPLETED\\
	(PI: Donald)	&	2003-2005 \\
	NSF	& EIA-0305444\\
{\em Algorithmic Challenges in Computational Biology}\\
\effortamount{        \$37,500 &\\}
\end{tabular}

This grant supported computational research in functional genomics and
computational methods in NMR structural biology.

%\myhrule

%\newpage

\myhrule

\noindent
\begin{tabular}{ll}
DONALD, BRUCE R.	&		COMPLETED\\
	(co-PI: B.~Donald)	&	1998-2003 \\
	NSF	& NSF 98-02068 \\
\end{tabular}

{\em Systems Science for Physical Geometric Algorithms}


NSF research infrastructure grant. The major goals of this project
were to provide research infrastructure for computational science and
computational biology in the Computer Science Department.  This
includes workstations, supercomputing facilities, networking, etc.

\myhrule

\newpage

\myhrule

\noindent
\begin{tabular}{ll}
DONALD, BRUCE R.	&		COMPLETED\\
	(PI: Donald)	&	2001-2003 \\
	NSF	& EIA-0102710\\
\end{tabular}

{\em Physical Geometric Algorithms and Systems for
        High-Throughput NMR Structural Biology}

  The major goals of this project were to
develop novel computational methods for biomolecular NMR.

\myhrule


\noindent
\begin{tabular}{ll}
DONALD, BRUCE R.	&		COMPLETED\\
	(PI: Donald)	&	2001-2003 \\
	NSF	& EIA-0102712\\
\end{tabular}

{\em Physical Geometric Algorithms and Systems for
        Structural Biology using Mass Spectrometry}

The major goals of this project were to
develop novel computational methods for structural mass
spectrometry and proteomics.

\myhrule

\noindent
\begin{tabular}{ll}
DONALD, BRUCE R.	&		COMPLETED\\
	(PI: Donald)	&	2000-2006 \\
	DHS/ODP	& 2000-DT-CX-K001
\end{tabular}

{\em Microelectromechanical Systems for Infosecurity}


The major goal of this project was the development of novel
microelectromechanical systems ("MEMS") to be useful in micro- and
nano-technology applications for homeland security, in particular,
information security and micro robotics.

\myhrule



\subsection*{PENDING SUPPORT}


\myhrule 
\noindent \begin{tabular}{ll} DONALD, BRUCE R.& PENDING \\
 (PI: D. Madden, Co-investigator: B. Donald) & 2007-2012\\
Submitted to NIH (NIDDK) & R01 Application \\
 {\em Keeping CFTR in its
Place: An Integrated Small-Molecule Approach}\\
\end{tabular}

The major goal of this project is to develop an integrated
experimental and theoretical approach to identifying small-molecule
inhibitors selective for the CAL PDZ domain (a molecular scaffolding
protein), which interacts with the cytoplasmic C-terminus of the
cystic fibrosis transmembrane conductance regulator protein (CFTR).
 

\myhrule 
\noindent \begin{tabular}{ll} DONALD, BRUCE R.& PENDING \\
 (PI: J. Hoch, Co-PIs: B. Donald, G. Wagner, A. Alexandrescu,
 P. Bolton) & 2006-2008\\ 
Submitted to NSF & MRI application \\
{\em	Acquisition of a High Perfomance Computational
Resource for NMR Structural Biology}.&\\
\end{tabular}

The major goal of this project is the acquisition of a supercomputer
that will be attached to a high field NMR spectrometer in order to
facilitate the development of data-directed high-throughput
computational protocols for NMR data processing, automated assignment,
and structure determination.

\myhrule 

\noindent \begin{tabular}{ll} DONALD, BRUCE R.& PENDING \\
 (PI: B. Donald) & 2007-2012\\
Submitted to NIH (NIGMS) & R01 Application \\
 {\em Computational Active-Site Redesign and Binding Prediction via Molecular Ensembles}\\
\end{tabular}

The major goal of this project is to develop novel algorithms to plan
structure-based site-directed mutations to a protein's active site in
order to modify its function.  The new algorithms will make progress
towards the long-term objective of reprogramming the specificity of
non-ribosomal peptide synthetase domains, whose products include
natural antibiotics, antifungals, antivirals, immunosuppressants, and
antineoplastics.


\myhrule 

 
\end{document}

%%% Local Variables:
%%% write-file-hooks:   (time-stamp)
%%% time-stamp-active:  t
%%% time-stamp-start:   "\\\\def\\\\mydate{"
%%% time-stamp-end:     "}"
%%% time-stamp-line-limit: 20
%%% End:
